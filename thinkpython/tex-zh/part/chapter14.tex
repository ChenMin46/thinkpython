\chapter{文件}

\index{文件}
\index{类型!文件}


\section{持久性}

\index{持久性}

目前我们所见到的大多数的程序都是瞬态的,即它们在短时间内运行并输出一些结果,当它们结束时,数据也就消失了。如果你再次运行程序,它将从一个初始的状态开始。

另一类程序被称为是{\bf 持久的},它们长时间运行(或者时刻运行),至少将一部分数据记录在永久储存设备(如硬盘)上,当程序关闭并重新启动时,它们可以恢复结束前的状态以便继续运行。

一个持久的程序的例子是操作系统,当一台电脑开机后操作系统在绝大多数时间都在运行,对于一个接受网络请求的web服务器,操作系统时刻在运行。

程序维护数据最简单的方法是读写文本文件。我们已经接触过读取文本文件的程序,在本章中我们将接触写入文本文件的程序。

记录程序状态的另一个方式是使用数据库。在本章节中我给出一个简单的数据库和模块{\tt pickle},该模块简化了存储数据的过程。

\index{pickle模块}
\index{模块!pickle}


\section{读取和写入}

\index{文件!读取和写入}

文本文件是储存在类似硬盘、闪存、CD-ROM等永久介质上的字符序列。我们在章节~\ref{单词表}中接触了文件的打开和读取。

\index{open函数}
\index{函数!open}

要写入一个文件,你需要在打开文件时添加第二个参数\verb"'w'":

\beforeverb
\begin{verbatim}
>>> fout = open('output.txt', 'w')
>>> print fout
<open file 'output.txt', mode 'w' at 0xb7eb2410>
\end{verbatim}
\afterverb
%
如果文件已经存在,以写的模式打开该文件将会清空原来的数据并从新的开始,所以要小心!如果文件不存在,那么将创建一个新的文件。

{\tt write}方法将数据写入文件。

\beforeverb
\begin{verbatim}
>>> line1 = "This here's the wattle,\n"
>>> fout.write(line1)
\end{verbatim}
\afterverb
%
文件对象将跟踪位置,如果你再次调用{\tt write},它将在尾部写入新的数据。

\beforeverb
\begin{verbatim}
>>> line2 = "the emblem of our land.\n"
>>> fout.write(line2)
\end{verbatim}
\afterverb
%
当你完成写入,你可以关闭文件。

\beforeverb
\begin{verbatim}
>>> fout.close()
\end{verbatim}
\afterverb
%

\index{close方法}
\index{方法!close}


\section{格式运算符}

\index{格式运算符}
\index{运算符!格式}

{\tt write}的参数必须是字符串,如果我们要将其他值写入文件,我们需要将它们转换为字符串。最简单的方法是使用{\tt str}:

\beforeverb
\begin{verbatim}
>>> x = 52
>>> f.write(str(x))
\end{verbatim}
\afterverb
%
另一个方法是使用{\bf 格式运算符}{\tt \%}。当作用于整数,{\tt \%}是取模运算符,而当第一个运算数是字符串时,{\tt \%}是格式运算符。

\index{格式字符串}

第一个运算数是{\bf 格式字符串},它包含一个或多个{\bf 格式序列},它们指定了第二个运算数是如何格式化的。结果为一个字符串。

\index{格式序列}

例如,格式序列\verb"'%d'"意味着第二个运算数应该被格式化为一个整数({\tt d}代表“decimal”):

\beforeverb
\begin{verbatim}
>>> camels = 42
>>> '%d' % camels
'42'
\end{verbatim}
\afterverb
%
结果是字符串\verb"'42'",需要和整数值{\tt 42}区分开来。

格式序列可以出现在字符串中的任何位置,所以你可以将值嵌入到语句中:

\beforeverb
\begin{verbatim}
>>> camels = 42
>>> 'I have spotted %d camels.' % camels
'I have spotted 42 camels.'
\end{verbatim}
\afterverb
%
如果字符串中有多于一个格式序列,第二个参数必须为一个元组。每个格式序列按次序和元组中的元素对应。

下面的例子中使用\verb"'%d'"来格式化一个整数。
\verb"'%g'" to format
a floating-point number (don't ask why), and \verb"'%s'" to format
a string:

\beforeverb
\begin{verbatim}
>>> 'In %d years I have spotted %g %s.' % (3, 0.1, 'camels')
'In 3 years I have spotted 0.1 camels.'
\end{verbatim}
\afterverb
%
元组中元素的个数必须等于字符串中格式序列的个数。同时,元素的类型必须符合对应的格式序列。

\index{异常!类型错误}
\index{类型错误}

\beforeverb
\begin{verbatim}
>>> '%d %d %d' % (1, 2)
TypeError: not enough arguments for format string
>>> '%d' % 'dollars'
TypeError: illegal argument type for built-in operation
\end{verbatim}
\afterverb
%
在第一个例子中,元组中没有足够的元素,在第二个例子中,元素的类型错误。

格式运算符十分强大,但它很难使用。你可以在\url{docs.python.org/lib/typesseq-strings.html}阅读更多有关的内容。


\section{文件名和路径}
\label{路径}

\index{文件名}
\index{路径}
\index{目录}
\index{文件夹}

文件以{\bf 目录} (也称为“文件夹”)的形式管理。每个运行的程序有一个“当前目录”,它是许多操作的默认目录。例如,当你打开一个文件来读取数据,Python在当前目录下寻找这个文件。

\index{os模块}
\index{模块!os}

{\tt os}模块提供了操作文件和目录的函数(“os”代表“operating system”)。{\tt os.getcwd}返回当前目录的名称:

\index{getcwd函数}
\index{函数!getcwd}

\beforeverb
\begin{verbatim}
>>> import os
>>> cwd = os.getcwd()
>>> print cwd
/home/dinsdale
\end{verbatim}
\afterverb
%
{\tt cwd}代表“current working directory”,即当前工作陌路。在本例中结果是{\tt /home/dinsdale},它是用户名为{\tt dinsdale}的主目录。

\index{工作目录}
\index{目录!工作}
类似{\tt cwd}能够确定一个文件的字符串称为{\bf 路径}。{\bf 相对路径}从当前目录开始,{\bf 绝对路径}从文件系统的根目录开始。

\index{相对路径}
\index{路径!相对}
\index{绝对路径}
\index{路径!绝对}

我们现在看到的路径都是简单的文件名,因此它们是相对当前目录的。要得到一个文件的绝对目录,你可以使用{\tt os.path.abspath}:

\beforeverb
\begin{verbatim}
>>> os.path.abspath('memo.txt')
'/home/dinsdale/memo.txt'
\end{verbatim}
\afterverb
%
{\tt os.path.exists}检查一个文件或者目录是否存在:

\index{exists函数}
\index{函数!exists}

\beforeverb
\begin{verbatim}
>>> os.path.exists('memo.txt')
True
\end{verbatim}
\afterverb
%
如果存在,{\tt os.path.isdir}检查它是否是一个目录:

\beforeverb
\begin{verbatim}
>>> os.path.isdir('memo.txt')
False
>>> os.path.isdir('music')
True
\end{verbatim}
\afterverb
%
类似的,{\tt os.path.isfile}检查是否是一个文件。

{\tt os.listdir}返回给定目录下的文件(以及其他目录):

\beforeverb
\begin{verbatim}
>>> os.listdir(cwd)
['music', 'photos', 'memo.txt']
\end{verbatim}
\afterverb
%
为了演示这些函数,下面的例子“遍历”一个目录,打印所有文件的名字,并对所有目录递归地调用自身。

\index{遍历,目录}
\index{目录!遍历}

\beforeverb
\begin{verbatim}
def walk(dir):
    for name in os.listdir(dir):
        path = os.path.join(dir, name)

        if os.path.isfile(path):
            print path
        else:
            walk(path)
\end{verbatim}
\afterverb
%
{\tt os.path.join}读取一个目录名和一个文件名,并将两者合并为一个完整路径。 
\begin{ex}
修改{\tt walk}函数,使之返回文件名列表,而不是打印这些信息。
\end{ex}

\begin{ex}
{\tt os}模块提供了与我们的{\tt walk}函数类似的函数,但功能更丰富。阅读文档,使用该函数打印给定目录下的文件和子目录。
\end{ex}


\section{捕获异常}
\label{捕获}
当你尝试读写文件时,很多地方会发生错误。如果你试图打开一个不存在的文件,你会得到一个{\tt 输入输出错误}:

\index{open海曙}
\index{函数!open}
\index{异常!输入输出错误}
\index{输入输出错误}

\beforeverb
\begin{verbatim}
>>> fin = open('bad_file')
IOError: [Errno 2] No such file or directory: 'bad_file'
\end{verbatim}
\afterverb
%
如果你没有权限访问一个文件:

\index{文件!权限}
\index{权限,文件}

\beforeverb
\begin{verbatim}
>>> fout = open('/etc/passwd', 'w')
IOError: [Errno 13] Permission denied: '/etc/passwd'
\end{verbatim}
\afterverb
%
如果你试图读取一个目录,你会得到:

\beforeverb
\begin{verbatim}
>>> fin = open('/home')
IOError: [Errno 21] Is a directory
\end{verbatim}
\afterverb
%
为了避免这些错误,你可以使用类似{\tt os.path.exists}和{\tt os.path.isfile}的检查函数,但是检查所有可能的错误会占用很多时间和代码(如果“{\tt Errno 21}”是一个错误信息,那么至少有21种出错情况)。

\index{异常,捕获}
\index{try语句}
\index{语句!try}

更好的方法是当问题出现了才去处理,即{\tt try}语句所做的。它的语法类似{\tt if}语句:

\beforeverb
\begin{verbatim}
try:    
    fin = open('bad_file')
    for line in fin:
        print line
    fin.close()
except:
    print 'Something went wrong.'
\end{verbatim}
\afterverb
%
Python从{\tt try}语句开始执行,如果一切正常,那么{\tt except}将被跳过。如果发生异常,则跳出{\tt try}语句块,执行{\tt except}中的代码。

使用{\tt try}语句处理异常被称为{\bf 捕获}异常。在本例中,{\tt except}语句块中的代码仅仅打印了错误信息。通常,捕获异常给了你修补问题的机会,你可以继续尝试,或者至少可以优雅的结束程序。


\section{数据库}

\index{数据库}

{\bf 数据库}是用于存储数据的文件。大多数的数据库以字典的形式组织,即将键映射为值。数据库是保存在磁盘(或其他永久存储设备)上,因此即使程序结束它们仍然存在。


\index{anydbm模块}
\index{模块!anydbm}

模块{\tt anydbm}提供了创建和跟新数据库文件的接口。作为一个例子,我将创建一个包含图片文件标题的数据库。

\index{open函数}
\index{函数!open}

打开数据库和其他文件类似:

\beforeverb
\begin{verbatim}
>>> import anydbm
>>> db = anydbm.open('captions.db', 'c')
\end{verbatim}
\afterverb
%
模式\verb"'c'"代表如果文件不存在则创建文件。返回结果是一个数据库对象,它可以像字典一样被使用(对于大多数的操作)。如果你创建一个新的项目, {\tt anydbm}将更新数据库文件。

\index{更新!数据库}


\beforeverb
\begin{verbatim}
>>> db['cleese.png'] = 'Photo of John Cleese.'
\end{verbatim}
\afterverb
%
当你访问某个项目是,{\tt anydbm}将读取文件:

\beforeverb
\begin{verbatim}
>>> print db['cleese.png']
Photo of John Cleese.
\end{verbatim}
\afterverb
%
如果你对已有的键再次进行赋值,{\tt anydbm}将替代旧的值:

\beforeverb
\begin{verbatim}
>>> db['cleese.png'] = 'Photo of John Cleese doing a silly walk.'
>>> print db['cleese.png']
Photo of John Cleese doing a silly walk.
\end{verbatim}
\afterverb
%
许多字典的方法,如{\tt keys}和{\tt items},同样适用于数据库对象,包括使用{\tt for}语句实现的迭代:

\index{字典方法!anydbm模块}

\beforeverb
\begin{verbatim}
for key in db:
    print key
\end{verbatim}
\afterverb
%
和其他文件一样,当你完成操作后你需要关闭文件:

\beforeverb
\begin{verbatim}
>>> db.close()
\end{verbatim}
\afterverb
%

\index{close方法}
\index{方法!close}


\section{Pickling}

\index{pickling}

{\tt anydbm}模块的一个限制在于键和值必须是字符串。如果你尝试使用其他数据类型,你会得到一个错误。

\index{pickle模块}
\index{模块!pickle}

{\tt pickle}模块可以解决这个问题。它能将任何类型的对象翻译成适合在数据库中储存的字符窜,同时能将字符串还原成对象。

{\tt pickle.dumps}读取一个对象作为参数,并返回一个表征字符串({\tt dumps}是“dump string”(转储字符串)的缩写):

\beforeverb
\begin{verbatim}
>>> import pickle
>>> t = [1, 2, 3]
>>> pickle.dumps(t)
'(lp0\nI1\naI2\naI3\na.'
\end{verbatim}
\afterverb
%
这个格式对人类读者来说不是很好理解,但是对{\tt pickle}来说很好解释。{\tt pickle.loads}(“载入字符串”)可以重建对象:

\beforeverb
\begin{verbatim}
>>> t1 = [1, 2, 3]
>>> s = pickle.dumps(t1)
>>> t2 = pickle.loads(s)
>>> print t2
[1, 2, 3]
\end{verbatim}
\afterverb
%
虽然新的对象和老的对象有相同的值,它们(通常)不是同一个对象:

\beforeverb
\begin{verbatim}
>>> t1 == t2
True
>>> t1 is t2
False
\end{verbatim}
\afterverb
%
换言之,pickling然后unpickling等效于复制一个对象。

你可以使用{\tt pickle}在数据库中存储一个非字符串对象。事实上,这个组合非常常用,并有一个已经封装好的模块{\tt shelve}。

\index{shelve模块}
\index{模块!shelve}


\begin{ex}

\index{回文集合}
\index{集合!回文}

如果你做了章节~\ref{回文}中的练习,修改你的方案,创建一个数据库,将列表中的单词映射为使用同样字母的单词的列表。

编写另一个程序,读取数据库并以人类适宜阅读的格式打印内容。
\end{ex}


\section{管道}

\index{shell}
\index{管道}

大多数的操作系统提供一个命令行的接口,称为{\bf shell}。shell通常提供浏览文件系统和启动程序的命令。例如,在Unix系统中,你可以使用{\tt cd}改变目录,使用{\tt ls}显示一个目录的内容,通过输入{\tt firefox}来启动一个网页浏览器。

\index{ls (Unix 命令)}
\index{Unix 命令!ls}

任何你在shell中可以启动的程序也可以在Python中通过使用{\bf 管道}来启动。一个管道是一个表示活动进程的对象。

例如,Unix命令{\tt ls -l}将以详细格式显示当前目录下的内容。你可以使用{\tt os.popen}来启动{\tt ls}:

\index{popen函数}
\index{函数!popen}

\beforeverb
\begin{verbatim}
>>> cmd = 'ls -l'
>>> fp = os.popen(cmd)
\end{verbatim}
\afterverb
%
参数是一个包含shell命令的字符串。返回值是一个对象,就像打开一个文件一样。你可以使用{\tt readline}每次从输出读取一样,或者使用{\tt read}一次读取所有内容:

\index{readline方法}
\index{方法!readline}
\index{read方法}
\index{方法!read}

\beforeverb
\begin{verbatim}
>>> res = fp.read()
\end{verbatim}
\afterverb
%
当你完成操作后,你像关闭一个文件一样关闭管道:

\index{close方法}
\index{方法!close}

\beforeverb
\begin{verbatim}
>>> stat = fp.close()
>>> print stat
None
\end{verbatim}
\afterverb
%
返回值是{\tt ls}进程的最终状态,{\tt None}表示它正常结束(没有错误)。

\index{文件!压缩}
\index{压缩!文件}
\index{Unix 命令!gunzip}
\index{gunzip (Unix命令)}

管道的一个常用用法是增量的读取一个压缩文件,即不需要一次解压整个文件。下面的函数读取一个压缩文件名作为参数,使用{\tt gunzip}来解压文件,并返回一个管道:

\beforeverb
\begin{verbatim}
def open_gunzip(filename):
    cmd = 'gunzip -c ' + filename
    fp = os.popen(cmd)
    return fp
\end{verbatim}
\afterverb
%
如果你每次从{\tt fp}读取一行,你不需要将解压的文件保存在内存或磁盘上。


\section{编写模块}
\label{模块}

\index{模块,编写}
\index{单词统计}

任何包含Python代码的文件可以作为模块被导入。例如,假设你有文件{\tt wc.py},内容如下:

\beforeverb
\begin{verbatim}
def linecount(filename):
    count = 0
    for line in open(filename):
        count += 1
    return count

print linecount('wc.py')
\end{verbatim}
\afterverb
%
如果你运行这个程序,它将读取自身,并打印行数,结果是7。你也可以这样导入模块:

\beforeverb
\begin{verbatim}
>>> import wc
7
\end{verbatim}
\afterverb
%
现在你有一个模块对象{\tt wc}:

\index{模块对象}
\index{对象!模块}

\beforeverb
\begin{verbatim}
>>> print wc
<module 'wc' from 'wc.py'>
\end{verbatim}
\afterverb
%
这里提供了一个称为\verb"linecount"的函数:

\beforeverb
\begin{verbatim}
>>> wc.linecount('wc.py')
7
\end{verbatim}
\afterverb
%
以上就是如何编写Python模块。

这个例子中唯一的问题在于当你导入模块后,最后的测试代码被执行。通常当你导入一个模块,它定义新的函数,但并不执行它们。

\index{import语句}
\index{语句!import}


作为模块的程序通常写成一下结构:

\beforeverb
\begin{verbatim}
if __name__ == '__main__':
    print linecount('wc.py')
\end{verbatim}
\afterverb
%
\verb"__name__"是一个程序开始时设置的内建变量。如果程序以脚本的形式运行,\verb"__name__"的值为\verb"__main__",在这种情况下,测试代码将被执行。否则,如果是作为模块被导入,测试代码将被忽略。

\begin{ex}
将例子输入到文件{\tt wc.py}中,并以脚本形式运行。如果在Python解释器中运行{\tt import wc},当模块被导入后,\verb"__name__"的值是什么?

警告:当你再次导入一个已经导入的的模块,Python将什么也不错。它不会重新读取文件,即使文件发生了改变。

\index{模块!重载}
\index{reload函数}
\index{函数!reload}
如果你要重载一个模块,你可以使用内建函数{\tt reload},但它可能会出错,因此最安全的方法是重启解释器然后重新导入模块。
\end{ex}



\section{调试}

\index{调试}
\index{空白}

当你读写文件,你也许会遇到空白带来的问题。由于空格符、tab符和换行符通常是不可见的,这样的错误很难调试:

\beforeverb
\begin{verbatim}
>>> s = '1 2\t 3\n 4'
>>> print s
1 2	 3
 4
\end{verbatim}
\afterverb

\index{repr函数}
\index{函数!repr}
\index{字符串表示}

内建函数{\tt repr}可以用来解决这个问题。它读取一个对象作为参数,并返回一个表示这个对象的字符串。对于字符串,它将空白符号用反斜杠序列表示:

\beforeverb
\begin{verbatim}
>>> print repr(s)
'1 2\t 3\n 4'
\end{verbatim}
\afterverb

这个对调试很有用。

你也许会遇到另一个问题,不同的操作系统使用不同的符号作为换行符。有的系统使用\verb"\n",有的使用\verb"\r",有的使用两者。如果你在不同系统中使用,这些差异会导致问题。

\index{行结束符号}
大多数系统提供了格式转换的程序。你可以在\url{wikipedia.org/wiki/Newline}中找到(并阅读更多相关内容)。当然你可以自己编写一个转换程序。


\section{术语}

\begin{description}

\item[持久性:] 一个长期运行、并至少将一部分数据保存在永久性的储存设备上的程序。
\index{持久性}

\item[格式运算符:] 运算符{\tt \%},参数为一个格式字符串和一个元组,生成一个按格式字符串规定的元组中元素的值的字符串。
\index{格式运算符}
\index{运算符!格式}

\item[格式字符串:] 使用格式运算符,包含格式序列的字符串。
\index{格式字符串}

\item[格式序列:] 格式字符串中的字符序列,类似{\tt \%d},指定了一个值的格式。
\index{格式序列}

\item[文本文件:] 保存在类似硬盘的永久储存设备上的字符序列。
\index{文本文件}

\item[目录:] 一个命名的文件的集合,也称为文件夹。
\index{目录}

\item[路径:] 用于识别文件的字符串。
\index{路径}

\item[相对路径:] 从当前目录开始的路径。
\index{相对路径}

\item[绝对路径:] 从文件系统顶部开始的路径。
\index{绝对路径}

\item[捕获:] 为了防止程序因为异常而终止,使用{\tt try}和{\tt except}语句来捕捉异常。
\index{捕获}

\item[数据库:] 一个类似字典使用键对应值的文件。
\index{数据库}

\end{description}


\section{练习}

\begin{ex}
\label{urllib}

\index{urllib模块}
\index{模块!urllib}
\index{URL}

{\tt urllib}模块提供了操作URL和从互联网下载信息的方法。下面的例子从{\tt thinkpython.com}下载并打印一条秘密信息:

\beforeverb
\begin{verbatim}
import urllib

conn = urllib.urlopen('http://thinkpython.com/secret.html')
for line in conn.fp:
    print line.strip()
\end{verbatim}
\afterverb

运行程序,运行结果将给你下一步指令。

\index{秘密练习}
\index{练习,秘密}

\end{ex}

\begin{ex}
\label{和校验}

\index{MP3}

在一个有很多MP3文件的收藏中,有可能同一首歌有多个拷贝,以不同的名字保存在不同的目录下。这个练习的目的是找出这些拷贝。

\begin{enumerate}

\item 编写程序,递归地搜索一个目录和所有子目录,返回一个完整路径的后缀给定的(如{\tt .mp3})的文件名列表。提示:{\tt os.path}提供了几个有用的操作文件和路径名的函数。

\index{拷贝}
\index{MD5算法}
\index{算法!MD5}
\index{和校验}

\item 为了识别拷贝,你可以使用哈希函数,读取文件并生成一个针对内容的简短的概述。例如,MD5 (Message-Digest algorithm 5)读取一个任意长的“消息”并返回一个128比特的“校验和”。两个不同文件返回相同的校验和的概率非常小。

你可以在\url{wikipedia.org/wiki/Md5}了解有关MD5的知识。在一个Unix系统上你可以使用{\tt md5sum}程序和Python中的管道来计算校验和。

\index{管道}

\end{enumerate}

\end{ex}


\begin{ex}

\index{网络电影数据库(IMDb)}
\index{IMDb (网络电影数据库)}
\index{数据库}

网络电影数据库(IMDb)是一个在线的电影信息收集的网站。它们的数据库可以以纯文本的格式获得,便于Python的读取。在这个练习中,你需要文件{\tt actors.list.gz}和{\tt actresses.list.gz},它们可以从\url{www.imdb.com/interfaces#plain}下载。

\index{纯文本}
\index{文本!纯}
\index{解析}

我编写了一个程序,可以解析这些文件并分割为演员名、电影标题等。你可以在\url{thinkpython.com/code/imdb.py}下载。

如果你已脚本方式运行{\tt imdb.py},程序将读取{\tt actors.list.gz}并在每行答应演员-电影对。或者你可以{\tt import imdb},调用函数\verb"process_file"来处理这些文件。参数为一个文件名,一个函数对象和一个可选的指定处理行数的数。下面给出一个例子:

\beforeverb
\begin{verbatim}
import imdb

def print_info(actor, date, title, role):
    print actor, date, title, role

imdb.process_file('actors.list.gz', print_info)
\end{verbatim}
\afterverb

当你调用\verb"process_file",它打开{\tt filename},读取内容,并对文件中的每行调用\verb"print_info"。\verb"print_info"读取演员、日期、电影标题和角色作为参数,并打印这些信息。

\begin{enumerate}

\item 编写函数,读取{\tt actors.list.gz}和{\tt actresses.list.gz},使用{\tt shelve}来构建一个数据库,将每个演员映射到他或她的电影列表。

\index{shelve模块}
\index{模块!shelve}

\item 两个演员称为是“合演”的,如果他们至少一起演过一部电影。基于上一步建立的数据库,构建第二个数据库,将每个演员映射到他或她“合演”的演员列表。

\index{Bacon, Kevin}
\index{Kevin Bacon Game}

 \item 编写程序,实现“Six Degrees of Kevin Bacon”,你可以在\url{wikipedia.org/wiki/Six_Degrees_of_Kevin_Bacon}中了解相关信息。这个问题的挑战在于你需要在关系图上找到最短路径。你可以在\url{wikipedia.org/wiki/Shortest_path_problem}里阅读最短路径算法相关的资料。

\end{enumerate}

\end{ex}
