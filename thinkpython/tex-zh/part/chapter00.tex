

%LaTeXOnly

\input{latexonly}

\newtheorem{ex}{Exercise}[chapter]
\begin{latexonly}

\renewcommand{\blankpage}{\thispagestyle{empty} \quad \newpage}

%blankpage
%blankpage

%-half title----------------------------------------------
\thispagestyle{empty}
\begin{flushright}
\vspace*{2.0in}

\begin{spacing}{3}
{\huge 思考Python}\\
{\Large 像计算机科学家一样思考}
\end{spacing}

\vspace{0.25in}

Version \theversion

\vfill
\end{flushright}

%---------------------------
\blankpage
\blankpage

%---title page---------
\pagebreak
\thispagestyle{empty}

\begin{flushright}
\vspace*{2.0in}

\begin{spacing}{3}
{\huge 思考Python}\\
{\Large 像计算机科学家一样思考}
\end{spacing}

\vspace{0.25in}
Version \theversion

{\Large
	Allen Downey\\
}
{\Large 
	Walter Lewis\\
}



\vspace{0.5in}

{\Large Green Tea Press}\\
{\small Needham, Massachusetts}

\vfill
\end{flushright}

%---copyright------------------
\pagebreak
\thispagestyle{empty}

{\small
	Copyright \copyright ~2008 Allen Downey.

Printing history:

\begin{description}

\item[2002四月:] 第一版 {\em 像计算机科学家一样思考}.
\item[2007八月:] 大幅改动,把标题改为{\em 像(Python)程序员一样思考}.
\item[2008六月:] 大幅改动,把标题改为{\em 思考Python:像计算机科学家一样思考}.
\end{description}

\vspace{0.2in}

\begin{flushleft}  %左对齐,类似的有flushright,centre
Green Tea Press \\
9 Washburn Ave\\
Needham MA 02492
\end{flushleft}


Permission is granted to copy, distribute, and/or modify this document
under the terms of the GNU Free Documentation License, Version 1.1 or
any later version published by the Free Software Foundation; with no
Invariant Sections, no Front-Cover Texts, and with no Back-Cover Texts.

The GNU Free Documentation License is available from {\tt www.gnu.org}
or by writing to the Free Software Foundation, Inc., 59 Temple Place,
Suite 330, Boston, MA 02111-1307, USA.

The original form of this book is \LaTeX\ source code.  Compiling this
\LaTeX\ source has the effect of generating a device-independent
representation of a textbook, which can be converted to other formats
and printed.

The \LaTeX\ source for this book is available from
\url{http://www.thinkpython.com}
\vspace{0.2in}
}

\end{latexonly}


%htmlonly
\begin{htmlonly}

%Title page for html version

{\Large \thetitle}
\{\Large  Allen B.Downet}
\{\Large  翻译:Walter Lewis}

Version \theversion

\setcounter{chapter}{-1}
\end{htmlonly}

\chapter{前言}

\section{本书的奇怪历史}

1999年一月份的时候,我准备用Java教一门介绍性的编程课。在那之前,我已经
教了三次,而且每次我都很失望。这门课的挂课率非常之高,尽管对那些通过的学
生来说,整体的水平也是很低的。\\

我认为问题的根源之一是教科书。教科书太厚了,掺杂着大量不必要的Java细节
内容,并且没有足够高水平的引导去指导学生如何编程。学生们深陷“陷阱门“:他们起步很轻松,逐步的学习,突然,大约在第五章的某个位置,困难出现了。学生必须快速的学习大量的新内容。结果,我不得不把剩下的学期花在挑选一些片段来教学。\\

课程开始的前两周,我决定自力更生--自己编写书。我的目标是:
\begin{itemize}

\item 尽量简短.对学生来说,阅读十页比阅读无十页要好。

\item 注意词汇量。我尽量减少使用术语,而且在使用前必须先定义。

\item 逐步学习。为了避免陷阱门,我把最难的部分分解成一系列的小步骤。

\item 把重心放在编程,而不是编程语言。我采用最少的有用的Java语言的语法,
忽略其他的。
\end{itemize}

我需要一个书名,所以我就临时地把它叫做《像计算机科学家一样思考》\\

第一版很粗糙,但是很成功。学生们很乐意看它,并且能很好理解我在课堂上讲的难点,趣点和让他们实践的内容(这个最重要).\\

我用GNU自由文档许可证发布了这本书,读者们可以自由的复制,修改,发布这本书。 \\

\index{GNU Free Documentation License}
\index{Free Documentation License,GNU}

 接下来发生的事儿极其的有趣。Jeff Elkner,居住在弗尼亚的高中老师,改变了我的书,把它翻译成了Python。他给我寄了份他翻译的副本,于是乎我就有了一段不寻常的学习Python的经历--通过阅读我自己的书。\\

 Jeff和我随后修订了这本书,加入了Chris Meyers提供的一个案例学习。在2001年,我们共同发布了《像计算机科学家一样思考:Python编程》,当然同样是用GNU自由文档许可证。通过Gree Tea Press,我出版了这本书,并且开始在亚马逊和大学书店卖纸质书。Gree Tea Press出版的书可以从这儿获得\url{greenteapress.com}\\

 2003年,我开始在Olin College教书。第一次,我开始教Python。和教授Java的情况相反,学生们不再陷入泥潭,学到了更多,参与了很多有趣的项目,越学越带劲。\\

 在过去的五年里,我一直继续完善这本书,改正错误,提过某些例子的质量,加入一些其他的材料,特别是练习。在2008年,我开始重写这本书---同时,剑桥大学出版社的编辑联系到了我,他想出版本书的下一板。美妙的时刻!\\

 结构就出现了现在的这本书,不过有了一个简洁的名字《思考Python》。变化有:
 \begin{itemize}

 \item 在每一章末尾加了点调试的部分。这些部分提供了发现和避免bug的通用技巧,也对Python的陷阱提出了警告。

 \item 删除了最后几章关于列表和树实现的内容。虽然,我万分不舍,但是考虑到和本书余下的部分不协调,只能忍痛割爱。

 \item 增加了一些案例学习---提供了练习,答案和相关讨论的大例子。一些东西是基于Swampy,这是我为了教学而设计的Python程序。
 Swampy,代码实例和部分答案可以从这儿获得\url{thinkpython.com}.

 \item 扩展了关于程序构建计划和基本的设计模式的讨论。

 \item Python运用的更加地道。虽然这本书仍然是讨论编程的,而不是Python本身,但是现在我不得不承认这本书深受Python浸染。
 
 \end{itemize}

  我希望读者们可以享受这本书,也希望帮助你学习程序设计和像计算机科学家一样思考,哪怕是一丁丁点儿。\\

Allen B. Downey\\
Needham MA\\

Allen Downey 是Olin College 大学计算机科学与技术系的副教授。



\section*{声明}

首先,也是最重要的,我要感谢Jeff Elkner,是他把我的Java书翻译成了Python,也由此开启了这项“工程“,也把我领进了我最爱的编程语言大门。\\

我也要感谢Chris Meyers,他贡献了《像计算机科学家一样思考》的部分内容。\\

感谢FSF制定的GNU自由文档许可证,使我和Jeff 和Chris 的合作成为可能。\\

\index{GNU Free Documentation License}
\index{Free Documentation License,GNU}

我也要感谢所以使用以前版本的学生和所有的贡献者,他们提供了宝贵的更正和建议2。\\

感谢我的妻子,Lisa为她在这本书上所花费的努力,还有Gree Tea Press,和其他的一切。

\section*{贡献者名单}

\index{贡献者}

More than 100 sharp-eyed and thoughtful readers have sent in
suggestions and corrections over the past few years.  Their
contributions, and enthusiasm for this project, have been a
huge help.

If you have a suggestion or correction, please send email to 
{\tt feedback@thinkpython.com}.  If I make a change based on your
feedback, I will add you to the contributor list
(unless you ask to be omitted).

If you include at least part of the sentence the
error appears in, that makes it easy for me to search.  Page and
section numbers are fine, too, but not quite as easy to work with.
Thanks!

\small

\begin{itemize}

\item Lloyd Hugh Allen sent in a correction to Section 8.4.

\item Yvon Boulianne sent in a correction of a semantic error in
Chapter 5.

\item Fred Bremmer submitted a correction in Section 2.1.

\item Jonah Cohen wrote the Perl scripts to convert the
LaTeX source for this book into beautiful HTML.

\item Michael Conlon sent in a grammar correction in Chapter 2
and an improvement in style in Chapter 1, and he initiated discussion
on the technical aspects of interpreters.

\item Benoit Girard sent in a
correction to a humorous mistake in Section 5.6.

\item Courtney Gleason and Katherine Smith wrote {\tt horsebet.py},
which was used as a case study in an earlier version of the book.  Their
program can now be found on the website.

\item Lee Harr submitted more corrections than we have room to list
here, and indeed he should be listed as one of the principal editors
of the text.

\item James Kaylin is a student using the text. He has submitted
numerous corrections.

\item David Kershaw fixed the broken {\tt catTwice} function in Section
3.10.

\item Eddie Lam has sent in numerous corrections to Chapters 
1, 2, and 3.
He also fixed the Makefile so that it creates an index the first time it is
run and helped us set up a versioning scheme.  

\item Man-Yong Lee sent in a correction to the example code in
Section 2.4.  

\item David Mayo pointed out that the word ``unconsciously"
in Chapter 1 needed
to be changed to ``subconsciously".

\item Chris McAloon sent in several corrections to Sections 3.9 and
3.10.

\item Matthew J. Moelter has been a long-time contributor who sent
in numerous corrections and suggestions to the book.  

\item Simon Dicon Montford reported a missing function definition and
several typos in Chapter 3.  He also found errors in the {\tt increment}
function in Chapter 13.

\item John Ouzts corrected the definition of ``return value"
in Chapter 3.

\item Kevin Parks sent in valuable comments and suggestions as to how
to improve the distribution of the book.

\item David Pool sent in a typo in the glossary of Chapter 1, as well
as kind words of encouragement.

\item Michael Schmitt sent in a correction to the chapter on files
and exceptions.

\item Robin Shaw pointed out an error in Section 13.1, where the
printTime function was used in an example without being defined.

\item Paul Sleigh found an error in Chapter 7 and a bug in Jonah Cohen's
Perl script that generates HTML from LaTeX.

\item Craig T. Snydal is testing the text in a course at Drew
University.  He has contributed several valuable suggestions and corrections.

\item Ian Thomas and his students are using the text in a programming
course.  They are the first ones to test the chapters in the latter half
of the book, and they have made numerous corrections and suggestions.

\item Keith Verheyden sent in a correction in Chapter 3.

\item Peter Winstanley let us know about a longstanding error in
our Latin in Chapter 3.

\item Chris Wrobel made corrections to the code in the chapter on
file I/O and exceptions. 

\item Moshe Zadka has made invaluable contributions to this project.
In addition to writing the first draft of the chapter on Dictionaries, he
provided continual guidance in the early stages of the book.

\item Christoph Zwerschke sent several corrections and
pedagogic suggestions, and explained the difference between {\em gleich}
and {\em selbe}.

\item James Mayer sent us a whole slew of spelling and
typographical errors, including two in the contributor list.

\item Hayden McAfee caught a potentially confusing inconsistency
between two examples.

\item Angel Arnal is part of an international team of translators
working on the Spanish version of the text.  He has also found several
errors in the English version.

\item Tauhidul Hoque and Lex Berezhny created the illustrations
in Chapter 1 and improved many of the other illustrations.

\item Dr. Michele Alzetta caught an error in Chapter 8 and sent
some interesting pedagogic comments and suggestions about Fibonacci
and Old Maid.

\item Andy Mitchell caught a typo in Chapter 1 and a broken example
in Chapter 2.

\item Kalin Harvey suggested a clarification in Chapter 7 and
caught some typos.

\item Christopher P. Smith caught several typos and helped us
update the book for Python 2.2.

\item David Hutchins caught a typo in the Foreword.

\item Gregor Lingl is teaching Python at a high school in Vienna,
Austria.  He is working on a German translation of the book,
and he caught a couple of bad errors in Chapter 5.

\item Julie Peters caught a typo in the Preface.

\item Florin Oprina sent in an improvement in {\tt makeTime},
a correction in {\tt printTime}, and a nice typo.

\item D.~J.~Webre suggested a clarification in Chapter 3.

\item Ken found a fistful of errors in Chapters 8, 9 and 11.

\item Ivo Wever caught a typo in Chapter 5 and suggested a clarification
in Chapter 3.

\item Curtis Yanko suggested a clarification in Chapter 2.

\item Ben Logan sent in a number of typos and problems with translating
the book into HTML.

\item Jason Armstrong saw the missing word in Chapter 2.

\item Louis Cordier noticed a spot in Chapter 16 where the code
didn't match the text.

\item Brian Cain suggested several clarifications in Chapters 2 and 3.

\item Rob Black sent in a passel of corrections, including some
changes for Python 2.2.

\item Jean-Philippe Rey at Ecole Centrale
Paris sent a number of patches, including some updates for Python 2.2
and other thoughtful improvements.

\item Jason Mader at George Washington University made a number
of useful suggestions and corrections.

\item Jan Gundtofte-Bruun reminded us that ``a error'' is an error.

\item Abel David and Alexis Dinno reminded us that the plural of
``matrix'' is ``matrices'', not ``matrixes''.  This error was in the
book for years, but two readers with the same initials reported it on
the same day.  Weird.

\item Charles Thayer encouraged us to get rid of the semi-colons
we had put at the ends of some statements and to clean up our
use of ``argument'' and ``parameter''.

\item Roger Sperberg pointed out a twisted piece of logic in Chapter 3.

\item Sam Bull pointed out a confusing paragraph in Chapter 2.

\item Andrew Cheung pointed out two instances of ``use before def.''

\item C. Corey Capel spotted the missing word in the Third Theorem
of Debugging and a typo in Chapter 4.

\item Alessandra helped clear up some Turtle confusion.

\item Wim Champagne found a brain-o in a dictionary example.

\item Douglas Wright pointed out a problem with floor division in
{\tt arc}.

\item Jared Spindor found some jetsam at the end of a sentence.

\item Lin Peiheng sent a number of very helpful suggestions.

\item Ray Hagtvedt sent in two errors and a not-quite-error.

\item Torsten H\"{u}bsch pointed out an inconsistency in Swampy.

\item Inga Petuhhov corrected an example in Chapter 14.

\item Arne Babenhauserheide sent several helpful corrections.

\item Mark E. Casida is is good at spotting repeated words.

\item Scott Tyler filled in a that was missing.  And then sent in
a heap of corrections.

\item Gordon Shephard sent in several corrections, all in separate
emails.

\item Andrew Turner {\tt spot}ted an error in Chapter 8.

\item Adam Hobart fixed a problem with floor division in {\tt arc}.

\item Daryl Hammond and Sarah Zimmerman pointed out that I served
up {\tt math.pi} too early.  And Zim spotted a typo.

\item George Sass found a bug in a Debugging section.

\item Brian Bingham suggested Exercise~\ref{exrotatepairs}.

\item Leah Engelbert-Fenton pointed out that I used {\tt tuple}
as a variable name, contrary to my own advice.  And then found
a bunch of typos and a ``use before def.''

\item Joe Funke spotted a typo.

\item Chao-chao Chen found an inconsistency in the Fibonacci example.

\item Jeff Paine knows the difference between space and spam.

\item Lubos Pintes sent in a typo.

\item Gregg Lind and Abigail Heithoff suggested Exercise~\ref{checksum}.

\item Max Hailperin has sent in a number of corrections and
  suggestions.  Max is one of the authors of the extraordinary {\em
    Concrete Abstractions}, which you might want to read when you are
  done with this book.

\item Chotipat Pornavalai found an error in an error message.

\item Stanislaw Antol sent a list of very helpful suggestions.

\item Eric Pashman sent a number of corrections for Chapters 4--11.

\item Miguel Azevedo found some typos.

\item Jianhua Liu sent in a long list of corrections.

\item Nick King found a missing word.

\item Martin Zuther sent a long list of suggestions.

\item Adam Zimmerman found an inconsistency in my instance
of an ``instance'' and several other errors.

\item Ratnakar Tiwari suggested a footnote explaining degenerate
triangles.

\item Anurag Goel suggested another solution for \verb"is_abecedarian"
and sent some additional corrections.  And he knows how to
spell Jane Austen.

\item Kelli Kratzer spotted one of the typos.

\item Mark Griffiths pointed out a confusing example in Chapter 3.

\item Roydan Ongie found an error in my Newton's method.

\item Patryk Wolowiec helped me with a problem in the HTML version.

\item Mark Chonofsky told me about a new keyword in Python 3.0.

\item Russell Coleman helped me with my geometry.

\item Wei Huang spotted several typographical errors.

\item Karen Barber spotted the the oldest typo in the book.

\item Nam Nguyen found a typo and pointed out that I used the Decorator
pattern but didn't mention it by name.

\item St\'{e}phane Morin sent in several corrections and suggestions.

\item Paul Stoop corrected a typo in \verb+uses_only+.

\item Eric Bronner pointed out a confusion in the discussion of the
order of operations.

\item Alexandros Gezerlis set a new standard for the number and
quality of suggestions he submitted.  We are deeply grateful!

% ENDCONTRIB

\end{itemize}


\normalsize
\clearemptydoublepage

\begin{latexonly}

\tableofcontents

\clearemptydoublepage

\end{latexonly}
