\chapter{类和方法}

\section{面向对象特点}

\index{object-oriented programming 面向对象编程}

Python是一门面向对象的编程语言,意味着它提供很多特点支持面向对象编程。

完整的定义面向对象编程不是一件容易的事,但是我们已经看到它的一些特点:

\begin{itemize}

\item 程序由对象定义和函数定义组成,大多数的计算都在操作对象过程中进行。

\item 每个对象定义都和现实世界的一些对象或者概念符合,操作对象的函数和现实世界对象的交互方式相一致。

\end{itemize}

比如,在\ref{time}章节的{\tt Time}类和人们记录日期的方式相同,我们
定义的函数和人们处理时间的方式相一致。类似地,{\tt Point}和{\tt Rectangle}类和数学意义上的点和矩形相一致。\\

迄今为止,我们没有很好的利用Python提供的特点进行面向对象编程。这些特点不是强制使用;大多数特点为我们已经解决的问题提供了可供选择的语法。但是
杂很多情况下,可替代的方案更精简准确地表示程序的结构。\\

比如,在{\tt Time}程序中,类定义和紧跟其后的函数定义没有什么明显的联系。
仔细看看,就会发现,每个函数都接受了至少一个{\tt Time}对象作为参数。

\index{method 方法}
\index{function 函数}

这个发现激发了方法的出现;方法就是和一个特定的类结合在一起的函数。
我们已经于遇到了字符串,列表,字典和元组的方法。这一章,我们为自定义
类型定义方法。

\index{syntax 语法}
\index{semantics 语义}

方法在语义上就是函数,但是有来嗯个语法上的不同:

\begin{itemize}

\item 方法定义在类体里,使得方法和类的关系更为明显。

\item 调用方法的语法和函数不同。
\end{itemize}

在接下来的几个部分里,我们把前两章的函数拿过来,把它们转换成方法。
转换的方法是机械的,仅仅跟着一系列的步骤做就行了。如果,你很熟连把一种形式转换成另外一种形式,你将会更好的选择你想要的方式。

\section{Printing objects}
\label{print_time}

\index{object!printing}

在\ref{time}章,我们定义了一个{\tt Time}类,在练习\ref{printtime},你写了一个函数\verb"print_time":

\beforeverb
\begin{verbatim}
class Time(object):
    """represents the time of day.
       attributes: hour, minute, second"""

def print_time(time):
    print '%.2d:%.2d:%.2d' % (time.hour, time.minute, time.second)
\end{verbatim}
\afterverb

调用这个函数,你必须给函数传递一个{\tt Time}对象作为参数:

\beforeverb
\begin{verbatim}
>>> start = Time()
>>> start.hour = 9
>>> start.minute = 45
>>> start.second = 00
>>> print_time(start)
09:45:00
\end{verbatim}
\afterverb

要把\verb"print_time"转换成方法,我们所要做的就是把函数定义移到类定义
里。注意缩进。

\index{indentation 缩进}

\beforeverb
\begin{verbatim}
class Time(object):
    def print_time(time):
        print '%.2d:%.2d:%.2d' % (time.hour, time.minute, time.second)
\end{verbatim}
\afterverb

现在有两种方式调用\verb"print_time"。地一种方式是使用函数语法(不常见):

\index{fucntion syntax 函数语法}
\index{dot notation 点记法}

\beforeverb
\begin{verbatim}
>>> Time.print_time(start)
09:45:00
\end{verbatim}
\afterverb

这种方法里,{\tt Time}是类名,\verb"print_time"是方法名。{\tt start}作为参数被传递。

第二种(更简洁)方式是使用方法语法:

\index{method syntax 方法语法}

\beforeverb
\begin{verbatim}
>>> start.print_time()
09:45:00
\end{verbatim}
\afterverb
在这种方式里,\verb"print_time"是方法名,{\tt start}是方法被调用的对象,叫做主体。就像句子的主语是句子的主体一样,方法调用的主体方法的主体。

\index{subject 主体}

在方法内部,主体赋给第一个参数,这个例子里是{\tt start}被赋给{\tt time}。

\index{self (parameter name)}
\index{parameter!self}

习惯上,方法的第一个参数是{\tt self},所以更常见的是把\verb"print_time"写成这样:

\beforeverb
\begin{verbatim}
class Time(object):
    def print_time(self):
        print '%.2d:%.2d:%.2d' % (self.hour, self.minute, self.second)
\end{verbatim}
\afterverb
%

这个习惯的原因是一个隐喻:

\index{metaphor, method invocation 隐喻,方法调用}

\begin{itemize}

\item 函数调用的语法,\verb"print_time(start)",意味着函数是主体。好像这样,“嗨,\verb"print_time" !这里有一个对象需要你输出。“

\item 在面向对象编程中,对象是主体。方法调用\verb"start.print_time()"是这样的,“嗨,{\tt start}! 请打印自己。"

\end{itemize}

这个方式上的转变更加的礼貌,但是,你看不出有什么更有利的地方,在我们
遇到的例子里,确实是这样的。但是,有时候,把函数的责任调到对象身上使得函数更加实用,也更容一维护和复用。

\begin{ex}
\label{convert}

把\verb"time_to_int"函数(\ref{prototype}部分)改成方法。把\verb"int_to_time"函数改成方法是不恰当的,至少我们清楚要调用对象。
\end{ex}

\section{另外一个例子}

\index{increment 增量}

下面是{\tt increment}(\ref{增量})被改写为方法的一个版本:

\beforeverb
\begin{verbatim}
# inside class Time:

    def increment(self, seconds):
        seconds += self.time_to_int()
        return int_to_time(seconds)
\end{verbatim}
\afterverb

这个版本假定\verb"time_to_int"(练习\ref{转换})也是被写成了方法。必须
要提一下,这个是一个纯函数,不是一个改变版。

下面的是调用{\tt increment}的方式:

\beforeverb
\begin{verbatim}
>>> start.print_time()
09:45:00
>>> end = start.increment(1337)
>>> end.print_time()
10:07:17
\end{verbatim}
\afterverb

主体{\tt start}被赋给第一个参数{\tt self}。参数{\tt 1337},被赋给第二个参数{\tt seconds}。

这个方法是令人迷惑的,特别是当发生错误时。比如,如果传递两个参数调用{\tt increment},
会得到:

\index{exception!TypeError}
\index{TypeError}

\beforeverb
\begin{verbatim}
>>> end = start.increment(1337, 460)
TypeError: increment() takes exactly 2 arguments (3 given)
\end{verbatim}
\afterverb
%

错误信息乍看起来是令人迷惑的,因为在括号里明明只有2个参数。但是主体
也是被认为是参数,所有一用有3个参数。

\section{更复杂的一个例子}

\verb"is_after"(联系\ref{is_after})稍微有点复杂,因为它接受两个Time对象作为参数。此时,习惯上把第一个参数当作哦{\tt self},第二个参数作为其他:

\index{other (parameter name)}
\index{patameter!other}

\beforeverb
\begin{verbatim}
# inside class Time:

    def is_after(self, other):
        return self.time_to_int() > other.time_to_int()
\end{verbatim}
\afterverb

使用这个方法,必须使用一个对象作为主体,另外一个对象作为参数:

\beforeverb
\begin{verbatim}
>>> end.is_after(start)
True
\end{verbatim}
\afterverb

有趣的是这个几乎完全可以像英语一样读:“end is after start?"

\section{初始方法}

\index{init method 初始方法}
\index{method!init}

初始方法(简称"initialization")是一个特殊的方法当一个对象实例化的时候。全名是\verb"__init__"(两个下划线,然后是{\tt init},最后又是两个下划线)。{\tt Time}类的初始方法可以这么写:

\beforeverb
\begin{verbatim}
# inside class Time:

    def __init__(self, hour=0, minute=0, second=0):
        self.hour = hour
        self.minute = minute
        self.second = second
\end{verbatim}
\afterverb
%

\verb"__init__"的参数和类的属性拥有相同的名称是很常见的。语句

\beforeverb
\begin{verbatim}
        self.hour = hour
\end{verbatim}
\afterverb

存储了{\tt hour}的值作为{\tt self}的属性。

\index{optional parameter 可选择参数}
\index{parameter!optional}
\index{default value 缺省值}
\index{override 覆盖}

参数是可选的,如果调用{\tt Time}。而不指定参数,就会得到一个缺省值。

\beforeverb
\begin{verbatim}
>>> time = Time()
>>> time.print_time()
00:00:00
\end{verbatim}
\afterverb

如果提供了一个参数,就会覆盖{\tt hour}:

\beforeverb
\begin{verbatim}
>>> time = Time (9)
>>> time.print_time()
09:00:00
\end{verbatim}
\afterverb
%
如果提供两个参数,就会覆盖{\tt hour}和{'\tt minute}。

\beforeverb
\begin{verbatim}
>>> time = Time(9, 45)
>>> time.print_time()
09:45:00
\end{verbatim}
\afterverb


如果提供三个参数,就会覆盖所有的三个缺省值。

\begin{ex}
\index{Point class Point类}
\index{class!Point}

为{\tt Point}类编写一个初始化方法,接受{\tt x}和{\tt y}作为可选择参数,并且把他们赋给对应的属性。
\end{ex}

\section{\_\_str\_\_方法}

\index{str method@\_\_str\_\_方法}
\index{method!\_\_str\_\_}

\verb"__str__"像\verb"__init__"一样,是一个特殊的方法,返回一个字符串化的对象。

\index{string representation}

比如,这里是Time 对象的一个{\tt str}方法:

\beforeverb
\begin{verbatim}
# inside class Time:

    def __str__(self):
        return '%.2d:%.2d:%.2d' % (self.hour, self.minute, self.second)
\end{verbatim}
\afterverb
%

当{\tt print}一个对象时,Python调用{\tt str}方法:

\index{print statement print语句}
\index{statement!print}

\beforeverb
\begin{verbatim}
>>> time = Time(9, 45)
>>> print time
09:45:00
\end{verbatim}
\afterverb

当我编写新类的时候,我几乎都要从编写\verb"__init__"(使得更容易实例化对象)和\verb"__str__"(使得调试更方便)方法开始。

\begin{ex}
为{\tt Point}编写一个{\tt str}方法。创建一个Point对象,并打印它。
\end{ex}

\section{运算符重载}
\label{operator overloading 运算符重载}

通过定义其他的特殊方法,可以指明用户自定义类型的操作符的行为。比如,你为{\tt Time}类定义了一个方法\verb"__add__"方法,就可以在Time对象间
使用{\tt +}运算符。

下面是可能的定义:

\index{add method add方法}
\index{method!add}

\beforeverb
\begin{verbatim}
# inside class Time:

    def __add__(self, other):
        seconds = self.time_to_int() + other.time_to_int()
        return int_to_time(seconds)
\end{verbatim}
\afterverb

下面演示的是如何使用它:

\beforeverb
\begin{verbatim}
>>> start = Time(9, 45)
>>> duration = Time(1, 35)
>>> print start + duration
11:20:00
\end{verbatim}
\afterverb

当在Time对象间应用{\tt +}运算符时,Python调用\verb"__add__"。当打印结果的时候,Python调用\verb"__str__"。安静的表面蕴藏着无限的内容!

\index{operator overloading 运算符重载}

改变运算符的行为,适应自定义类型叫做运算符重载。Python中的每一个运算符都有一个对应的特殊方法,像\verb"__add__"。欲之详情参阅\url{docs.python.org/ref/specialnames.html}.

\begin{ex}
为Point类编写一个{\tt add}方法。
\end{ex}

\section{基于类型的调度}

前一部分,我们相加了两个对象,但我们也想把一个整数加到Time对象上。下下面是\verb"__add__"的一个版本,检查{\tt other}的类型,然后
调用\verb"add_time"或{\tt increment}:

\beforeverb
\begin{verbatim}
# inside class Time:

    def __add__(self, other):
        if isinstance(other, Time):
            return self.add_time(other)
        else:
            return self.increment(other)

    def add_time(self, other):
        seconds = self.time_to_int() + other.time_to_int()
        return int_to_time(seconds)

    def increment(self, seconds):
        seconds += self.time_to_int()
        return int_to_time(seconds)
\end{verbatim}
\afterverb

内置函数{\tt isinstance}接受一个值和一个类对象,如果值是类的对象,返回在{\tt True}。

\index{isinstance function isinstance函数}
\index{function!isinstance}

如果{\tt other}是一个时间对象,\verb"__add__"调用\verb"add_time",否则,函数认为参数是一个整数,调用{\tt increment}。这种操作叫做基于类型的调度,因为程序依据参数的类型分派不同的操作。

\index{type-based dispathch 基于类型的调度}
\index{dispatch ,type-based}

下面是一个使用{\tt +}运算符的例子,传递了不同类型的参数。

\beforeverb
\begin{verbatim}
>>> start = Time(9, 45)
>>> duration = Time(1, 35)
>>> print start + duration
11:20:00
>>> print start + 1337
10:07:17
\end{verbatim}
\afterverb
%

不幸的是,这个加法的实现不是可交换的,如果第一个操作数是整数,会得到:
\index{commutativity 可交换}

\beforeverb
\begin{verbatim}
>>> print 1337 + start
TypeError: unsupported operand type(s) for +: 'int' and 'instance'
\end{verbatim}
\afterverb


问题在于,不是要求Time对象去加一个整数,Python要求整数加一个Time对象
,当然,Python不知道如何做。有一个很巧妙的解决办法:特殊方法\verb"__radd__",代表"右边相加"。当Time对象出现在{\tt +}的右边时,这个方法被调用。下面是定义:

\index{radd method radd方法}
\index{method!radd}

\beforeverb
\begin{verbatim}
# inside class Time:

    def __radd__(self, other):
        return self.__add__(other)
\end{verbatim}
\afterverb
%
下面是使用方式:

\beforeverb
\begin{verbatim}
>>> print 1337 + start
10:07:17
\end{verbatim}
\afterverb
%

\begin{ex}

为Point类编写一个函数{\tt add},要求可以适用于操作数是Point对象或者是元组:

\begin{itemize}

\item 如果第二个操作数是点,方法返回新的点,坐标$X$等于操作数的$X$坐标之和,$y$也是。

\item 如果第二个操作数是元组,方法把元组的第一个元素加到$X$上,第二个加到$y$上,返回新的点。

\end{itemize}
\end{ex}

\section{多态}

基于类型的调度当必要时是非常有用的,但是并不总是很有必要。通常不必要
根据不同的类型编写不同的函数。

\index{type-based dispatch 基于类型的调度}
\index{dispatch!type-based}

我们为字符串编写的很多函数都是适用于任何线性数据结构。比如,
在\ref{histogram}部分,我们使用{\tt histogram}统计每个字符在单词中出现的次数。

\beforeverb
\begin{verbatim}
def histogram(s):
    d = dict()
    for c in s:
        if c not in d:
            d[c] = 1
        else:
            d[c] = d[c]+1
    return d
\end{verbatim}
\afterverb

这个函数也适用于列表,元组甚至是字典,只要{\tt s}的元素是散乱的,他们就可以作为{\tt d}的关键字\footnote{译注:这句话有待商榷,关键字必须是unmutable}。

\beforeverb
\begin{verbatim}
>>> t = ['spam', 'egg', 'spam', 'spam', 'bacon', 'spam']
>>> histogram(t)
{'bacon': 1, 'egg': 1, 'spam': 4}
\end{verbatim}
\afterverb

能够适用于不同类型的函数叫做多态。多态促进了代码的复用。比如,内置函数{\tt sum},求序列元素的和,对于元素支持相加的序列都是适用的。

\index{polymorphism 多态}

Time对象也提供了{\tt add}方法,所以{\tt sum}函数也使用。

\beforeverb
\begin{verbatim}
>>> t1 = Time(7, 43)
>>> t2 = Time(7, 41)
>>> t3 = Time(7, 37)
>>> total = sum([t1, t2, t3])
>>> print total
23:01:00
\end{verbatim}
\afterverb

总的来说,如果函数里定义的操作适用于给定的类型,那么函数就使用于那种类型。

最好的多态就是不自主的而产生的---你发现你写的函数适用于你没有打算适用的类型!

\section{调试}
\index{debugging 调试}

在程序运行时,给对象增加属性是合法的,但是如果你是一个类型论的坚持者,
使相同类型的对象拥有不同的属性是不可靠的习惯。最好的是在初始化方法里初始化所有的对象属性。

\index{init method init方法}
\index{attribute!initializating}


如果不确定对象时候拥有一个特定的属性,可以使用内置函数{\tt hasattr}(参看\ref{hasattr})。

\index{hasattr function hasattr函数}
\index{function!hasattr}
\index{dict attribute@\_\_dict\_\_ attribute}
\index{attribute!\_\_dict\_\_}

另外一种访问对象属性的方式是通过\verb"__dict__"方法,以字典的形式显示属性名(作为字符串)和值。

\beforeverb
\begin{verbatim}
>>> p = Point(3, 4)
>>> print p.__dict__
{'y': 4, 'x': 3}
\end{verbatim}
\afterverb
%

从调试的角度看,经常使用它是个不错的调试方式。

\beforeverb
\begin{verbatim}
def print_attributes(obj):
    for attr in obj.__dict__:
        print attr, getattr(obj, attr)
\end{verbatim}
\afterverb

\verb"print_attribute"遍历对象字典的所有项,打印对应的名称及其值。

\index{traversal!dictionary}
\index{dictionary!traversal}

内置函数{\tt getattr},接受一个对象和属性名,返回属性的值。

\index{getattr function getattr函数}
\index{function!getattr}

\section{术语表}

\begin{description}

\item [object-oriented language 面向对象语言:] 提供用户自定义类型和
方法语法等特点的语言,有利于面向对象编程。
\index{object-oriented language 面向对象语言}

\item [object-oriented programming 面向对象编程:]一种编程风格,倡导
数据和操作封装在类及其方法里。
\index{object-oriented programming 面向对象编程}

\item [method 方法:] 定义在类里的函数,通过类的实例调用。
\index{method 方法}

\item [subject 主体:]方法被调用的对象。
\index{subject 主体}

\item [operator overloading 运算符重载:]改变像{\tt +}之类运算符的行为,使得适用于用户自定义类型。
\index{overloading 重载}
\index{operator!overloading}

\item [type-based dispatch 基于类型的调度:]一种编程模式,检查操作数
类型,然后调用不同的函数。
\index{type-based dispatch 基于类型的调度}

\item[polymorphism 多态:] 函数适用于不同类型的数据。

\index{polymorphism 多态}

\end{description}

\section{练习}

\begin{ex}

\index{default value!avouding mutable}
\index{mutable object, as default value}
\index{worst bug 最糟糕的bug}
\index{bug!worst}

这个练习是对Python中最常见也是最难发现的错误之一的一个警告。

\begin{enumerate}

\index{Kangaroo class Kangaroo类}
\index{class!Kangaroo}

\item 编写一个类{\tt Kangaroo},要求拥有下面的方法:

\begin{enumerate}

\item \verb"__init__"方法,初始化一个属性\verb"pouch_contents"为空列表。

\item \verb"put_in_pouch",接受一个任意类型的对象,加到\verb"pouch_contens"上。

\item \verb"__str__"方法,返回字符串化的Kangaroo对象和袋鼠袋子里的东西。

\end{enumerate}

测试你的代码:
创建两个{\tt Kangaroo}对象,分别赋给{\tt kanga}和{\tt roo},把{\tt roo}加到{\tt kanga}的袋子里。

\item 下载\url{thinkpython.com/code/BadKangaroo.py}。解答里包含了前一个
问题的解答,但是有一个很大的bug。找出并改正它。

如果你卡壳了,可以下载\url{thinkpython.com/code/GoodKangaroo.py},它解释了问题的所在并且给出了完整的解答。

\index{aliasing 别名}
\index{embedded object 嵌套对象}
\index{object!embedded}
\end{enumerate}
\end{ex}


\begin{ex}

\index{Visual module}
\index{module!Visual}
\index{vpython module}
\index{module!vpython}

Visual是Python的一个模块,提供了3D图形。通常在安装Python的时候,默认
是不安装它的,你可以从软件仓库里下载,如果那里没有,从这儿\url{vpython.org}。

下面的例子,创建了一个3D空间,宽,长,高均为256单元,中心被设定在 点
$(128,128,128)$,绘制了一个蓝色的地球。

\beforeverb
\begin{verbatim}
from visual import *

scene.range = (256, 256, 256)
scene.center = (128, 128, 128)

color = (0.1, 0.1, 0.9)          # mostly blue
sphere(pos=scene.center, radius=128, color=color)
\end{verbatim}
\afterverb

{\tt color}是RGB元组,也就是是哦,元组的元素是从0.0--1.0的RGB值(参看
\url{wikipedia.org/wiki/RGB_color_model})。

如果运行这个代码,你将会看到一个黑色背景和蓝色地球的窗口。如果上下拖拉中间的按钮,可以放大缩小图像。也拖拉右边的按钮旋转图形,但是只能显示一个地球,很难找出区别:

\beforeverb
\begin{verbatim}
t = range(0, 256, 51)
for x in t:
    for y in t:
        for z in t:
            pos = x, y, z
            sphere(pos=pos, radius=10, color=color)
\end{verbatim}
\afterverb


\begin{enumerate}

\item 把代码存储在脚本里,确保能够运行。

\item 修改程序,使得立方里的每个地球的颜色和它的位置相对应。注意:
坐标的范围是0--255,RGB元组的范围是0.0--1.0。

\index{color list}
\index{available colors}

\item 下载\url{thinkpython.com/code/color_list.py}。使用汉说
\verb"read_colors",产生你系统上可使用的颜色列表---颜色名和对应的RGB值。对于每一个颜色,在与RGB值对应的位置绘制一个地球。

\end{enumerate}

可以参看我的解答\url{thinkpython.com/code/color_space.py}.

\end{ex}





