\chapter{实例学习:字符处理}

\section{读取单词表}
\label{wordlist}

我们本章的联系需要一个英语单词表。在网络上有数以万计的单词表,但是
最适合我们的一个是贡献给公共域的单词表,它是由Grady Ward搜集整理作为Moby词典工程的一部分\footnote{\url{wikipedia.org/wiki/Moby_Project}.}。
它由113,809个官方纵横组合字谜的单词组成,也就是在纵横组合字谜中和其他字谜游戏中存在的单词组成。在Moby的搜集中,文件名是{\tt 113089.fic};我
复制了这个文件,命名为{\tt words.txt},并把它包含在了Swampy里了。

\index{Swampy}
\index{crosswords 纵横组合字谜}


这个文件是纯文本文件,你可以用一个编辑器打开它,当然,你也可以用python
来读取它。内置函数{\tt open}接受一个文件名作为参数,返回一个文件对象,
用它可以读取文件。

\index{oepn function open函数}
\index{function!open}
\index{plain text 纯文本文件}
\index{text!plain}
\index{object!file}
\index{file object  文件对象}

\beforeverb
\begin{verbatim}
>>> fin = open('words.txt')
>>> print fin
<open file 'words.txt', mode 'r' at 0xb7f4b380>
\end{verbatim}
\afterverb

{\tt fin}是赋给用来输入的文件对象的常见名称。大开模式\verb"'r'"表明
文件以只读方式大开(和\verb"'w'"以只写模式打开相反)。

\index{redline method readline方法}
\index{methods!readline}

文件对象提供了多种方式来读取文件,其中就包括{\tt readline},该函数从
文件中读取字符知道遇到换行符,然后以字符串的形式返回结果:

\beforeverb
\begin{verbatim}
>>> fin.readline()
'aa\r\n'
\end{verbatim}
\afterverb
%

这个单词表的第一个单词是"aa,"是一种熔岩的名称。序列\verb"\r\n"代表两个
空白字符,回车和换行,它们把这个单词下一行分隔开来。

文件对象自己跟踪达到了文件的什么地方,所以当再次调用{\tt readline}函数时,就会得到下一个单词:

\beforeverb
\begin{verbatim}
>>> fin.readline()
'aah\r\n'
\end{verbatim}
\afterverb

下一个单词是"aah,"---绝对合法的一个单词,不要以那么怪的眼神看我\footnote{译者:我没有笑话你的意思哦}。如果那个whitespace看上去很不给力,我们
使用{\tt strip}函数剔除它:

\index{strip method strip方法}
\index{method!strip}

\beforeverb
\begin{verbatim}
>>> line = fin.readline()
>>> word = line.strip()
>>> print word
aahed
\end{verbatim}
\afterverb

也可以在{\tt for}循环中使用文件对象。下面的程序对去{\tt word.txt},打印每一个单词,一行一个:

\index{open function open函数}
\index{function!open}

\beforeverb
\begin{verbatim}
fin = open('words.txt')
for line in fin:
    word = line.strip()
    print word
\end{verbatim}
\afterverb

\begin{ex}
编写一个程序,读取{\tt words.txt},打印包含超过20个字符的单词(不包括空白)。

\index{whitespace 空格}

\end{ex}

\section{练习}

下个部分有这些练习的答案。你至少得在参看答案之前尝试做一做每一道题。

\begin{ex}

1939年,Ernest Vincent Wright发表了一本50,000字的小说{\em Gadsby},在这本书中
不包含字母“e“。"e"在英语中是非常常见的字母,所以这件事很难做到。

事实上,如果不使用最常见的符号是很难想象那样的情况的。开始进展比较缓慢,但是
谨慎地训练几个小时之后,你就会逐步的掌握要领。

好的,进入正题!

编写一个函数\verb"has_no_e",如果给定的单词不含有字母"e",返回{\tt True}。

修改上一部分的程序,打印不含有"e"的单词,并计算不含有"e"的单词的百分比。

\index{lipogram 避讳}

\end{ex}

\begin{ex}

编写一个函数{\tt avoids}接受一个单词和一串“禁止”字母,如果单词没有使用禁止
字母中的任何一个,返回{\tt True}

修改你的程序提示用户输入一串禁止字母,然后打印不含有它们中任意一个的单词数目。你能找出由5个“禁止”字母组成的单词,不包括最小数目的单词吗?

\end{ex}

\begin{ex}
编写一个函数\verb"uses_only",接受一个单词和一串字母,如果单词仅仅包含列表
字符串里的字母,返回{\tt True}。除了"Hoe alfalfa"你能用{\tt acefhlo}里的字母
造一个句子吗?
\end{ex}

\begin{ex}

编写一个函数\verb"uses_all",接受一个单词和一串字母,如果单词使用一串字母里
的所有字母(至少一次),返回{\tt True}。有多少单词包含了全部的元音{\tt aeiou}?{\tt aeiouy}呢?
\end{ex}


\begin{ex}
编写一个函数\verb"is_abecedarian",如果单词里的字母以字典顺序出现,返回{\tt True}。有多少{\tt abecedarian}式的单词?
\end{ex}

\index{abecedarian}

\section{搜索}

\index{search pattern 搜索模式}
\index{pattern!search}






前一部分的所有联系都有一个共同点:他们都可以通过一个可搜索模式来解决,我们
在\ref{find}部分遇到过。最简单的例子是:

\beforeverb
\begin{verbatim}
def has_no_e(word):
    for letter in word:
        if letter == 'e':
            return False
    return True
\end{verbatim}
\afterverb

{\tt for}循环遍历{\tt word}中的每一个字母;如果不匹配,就进入下一个字母。
如果我们正常退出循环,就意味着我们没有找到“e“,于是返回{\tt True}。

\index{traversal 遍历}

\index{generalization}

{\tt avoids}是一个更一般的\verb"has_no_e"版本,但是有着相同的结构:

\beforeverb
\begin{verbatim}
def avoids(word, forbidden):
    for letter in word:
        if letter in forbidden:
            return False
    return True
\end{verbatim}
\afterverb

如果发现一个“禁止”字母,我们就返回{\tt False};如果我们达到了循环的结尾,
就返回{\tt True}。

\verb"uses_only"和它类似,出了条件的意思是相反的。

\beforeverb
\begin{verbatim}
def uses_only(word, available):
    for letter in word: 
        if letter not in available:
            return False
    return True
\end{verbatim}
\afterverb

除了使用一串“禁止”字母,我们也可以使用可获得字母(avaliable)。如果我们在、
{\tt word}中发现一个字母不再可获得字母中,就返回{\tt False}。

\verb"uses_all"函数也类似,出了我们掉换了word 和一串字母的角色。
\beforeverb
\begin{verbatim}
def uses_all(word, required):
    for letter in required: 
        if letter not in word:
            return False
    return True
\end{verbatim}
\afterverb

我们没有遍历{\tt word}中的字母,循环遍历了“要求“的字母,如果任意一个“要求”
字母没有出现在{\tt word}中,我们返回{\tt False}。

\index{traversal 遍历}

如果你想像一个计算机科学家一样思考,你可能已经认出\verb"uses_all"是前面已经
解决过的问题的实例,你就会编写:

\beforeverb
\begin{verbatim}
def uses_all(word, required):
    return uses_only(required, word)
\end{verbatim}
\afterverb

这是程序设计方法中的一个实例---叫做问题识别,含义是你认识到现在解决的问题
是已经解决问题的一个实例,然后修改应用以前的解法。

\index{problem recognition 问题识别}
\index{development plan!problem recognition}

\section{使用索引循环}

\index{looping!with indices}
\index{index!looping with}

我用{\tt for}循环编写以前的函数,因为我只需要用到字符串里的字符;我没有必要
使用索引来解决问题。

对于\verb"is_abecedarian"我们必须要比较相邻的字母,{\tt for}循环就有点力不
从心了。




\beforeverb
\begin{verbatim}
def is_abecedarian(word):
    previous = word[0]
    for c in word:
        if c < previous:
            return False
        previous = c
    return True
\end{verbatim}
\afterverb

另外一种方法就是使用递归:

\beforeverb
\begin{verbatim}
def is_abecedarian(word):
    if len(word) <= 1:
        return True
    if word[0] > word[1]:
        return False
    return is_abecedarian(word[1:])
\end{verbatim}
\afterverb

也还可以使用{\tt while}循环:

\beforeverb
\begin{verbatim}
def is_abecedarian(word):
    i = 0
    while i < len(word)-1:
        if word[i+1] < word[i]:
            return False
        i = i+1
    return True
\end{verbatim}
\afterverb


循环从{\tt i = 0}开始,以{\tt i = len(word) -1}收尾。每次循环时,比较第$i$
(你可以把它当成当前的字符)和$i+1$字符(你可以把它当作第二个字符)。

如果下一个字符小于当前的字符(字典顺序是在前),字母次序就被打断了,我们
返回{\tt False}。

如果我们到达了循环的末尾并且没有发现错误,这个{\tt word}“通过“了测试。
为了让自己信服循环正确的结束了,考虑这样一个例子\verb"'flossy'"。这个单词
的长度是6,所以最后一次循环执行的时候{\tt i} 是4---正好是倒数第二个字符的索引。
在最后一次迭代中,程序比较倒数第二个字符和最后一个字符---这就是我们希望的~。

\index{palindrome 回文}

下面是一个\verb"is_palindrome"的版本(参看练习\ref{回文}) ,函数使用了两个
索引;一个从头部开始,逐步递增,另外一个从尾部开始,逐步递减。


\beforeverb
\begin{verbatim}
def is_palindrome(word):
    i = 0
    j = len(word)-1

    while i<j:
        if word[i] != word[j]:
            return False
        i = i+1
        j = j-1

    return True
\end{verbatim}
\afterverb

或者,你也注意到了这又是已解决问题的一个实例,你或许会这么写:

\beforeverb
\begin{verbatim}
def is_palindrome(word):
    return is_reverse(word, word)
\end{verbatim}
\afterverb

\index{problem recognition 问题识别}
\index{development plan!problem recognition}

我认为你做了练习\ref{is_reverse}。

\section{调试}

\index{debugging 调试}
\index{testing!is hard}
\index{program testing 程序测试}

测试程序是一种“折磨”。本章的函数相对来说比较容易测试,因为你可以手动的检查
输出结果。尽管如此,某些地方还是很难甚至不可能选择一个合适的单词表来测试
所有可能的错误。\\


拿\verb"has_no_e"做例子,有两种明显的情况需要测试:含有"e"的单词应该返回{\tt False};不含有的返回{\tt True}。对于这两种情况,应该都没有任何错误。\\


在这两中情况下又有一些不太明显的子情况。在含有“e“的单词中,我们应该测试
"e"在单词前,在单词尾,在中间的某个地方。也应该测试长单词,短单词,和非常
短的单词,像空字符串。空字符串是特殊情况中的一个例子,也是非常不明显的情况,而错误恰恰经常藏身此处。

\index{special case 特别情况}

除了测试你自己想出的情况,你也可以用单词表来测试,比如使用{\tt word.txt}。
通过查看输出,你就可以抓住错误,但是小心:你或许能抓住一种错误(本不该包括
的单词被包括了)但不是另外一种(本该包含的单词没有被包含)。

一般说来,测试能帮助我们发现bugs,但是产生好的测试案例不是一件容易的事,
而且即使你测试了,你也不能保证你的程序就是正确的。

\index{testing!and absence of bugs}

一位传奇式的计算机科学家这么说过:

\begin{quote}
程序测试能够发现bugs是存在的,但是永远发现不了bugs不存在。

---Edsger W.Dijkstra
\end{quote}

\index{Dijkstra, Edsger}

\section{术语表}

\begin{description}

\item [file object 文件对象:] 代表打开文件的数值。
\index{file object 文件对象}
\index{object!file}

\item [problem recognition 问题识别:]通过把问题阐释成已经解决问题的一个
实例来解决问题的方法。
\index{problem recognition 问题识别}

\item [special case 特殊情况:]非典型的或不明显的测试例子(很难完美解决)。
\index{special case 特殊情况}

\end{description}

\section{练习}

\begin{ex}


\index{Car Talk 汽车谈话}
\index{Puzzler 难题}
\index{double letters }

这个问题是基于广播节目{\em Car Talk}播发的一个难题\footnote{\url{www.cartalk.com/content/puzzler/transcripts/200725}.}:

\begin{quote}

 给我一个含有三个连续的双字母单词,我能给你几个几乎符合条件的单词,但是不是
 完全符合。比如,单词{\tt committee},c-o-m-m-i-t-t-e-e。如果没有了$i$在那里,会更完美;Mississippi也是:M-i-s-s-i-s-s-i-p-p-i。如果能够去掉这些$i$,
 结果就很完美。有一个单词有三个连续的双字母,据我所知,只有这一个。当然,
 也有可能有500多个,但是我只能想到这一个。那么这个单词是什么?
 \end{quote}

 编写一个程序,寻找它。你可以参看我的答案\url{thinkpython.com/code/cartalk.py}.

\end{ex}



\begin{ex}
下面的也是一个{\em Car Talk}难题\footnote{\url{www.cartalk.com/content/puzzler/transcripts/200803}.}:

\index{Car Talk}
\index{Puzzler 难题}
\index{odometer 里程表}
\index{palindrome 回文}


\begin{quote}
“有一天,我驾车在高速公路上行使,我偶然看到了我的里程表。像大多数的里程表
一样,它显示了六个数字,全部是里数。比如,如果我的小汽车行使额300,000里,
我将会看到3-0-0-0-0-0。\\

“现在,我那天看到的却是非常的有意思。我注意到后四个数字是回文的;也就是说
从前往后读和从后往前读是一样的。比如,5-4-4-5是回文,所以我的里程表可能显示3-1-5-4-4-5。\\

“行使了一里之后,后五位数是回文的了。比如,可能是3-6-5-4-5-6。再行使一里后
中间的四位又是回文的了。你或许猜到了,一公里后,六位数字是回文的了!\\

“问题是,当我第一次看里程表的时候,显示的是什么?”
\end{quote}

编写一个Python程序,测试所有的六位数,打印任何满足条件的数字。你可以参看我的答案\url{thinkpython.com/code/cartalk.py}.

\end{ex}

\begin{ex}

下面的又是一个{\em Car Talk}难题,你可以用搜索来解决它\footnote{\url{www.cartalk.com/content/puzzler/transcripts/200813}}:

\index{Car Talk}
\index{Puzzler  难题}
\index{palindrome 回文}

\begin{quote}
“最近,我拜访了妈妈。我们认识到组成我年龄的两位数倒过来时是她的年龄。
比如,如果她是73,我就是37。我们想知道这种事发生的频率是多少?但是我们
转移到了另外一个话题,我们也就没有得到答案。\\

“当我会到家的时候,我计算出组成我们年龄的两位数已经有六次是像上面那样
了。我也计算出,如果幸运的话,几年后又会发生了,如果我们真的幸运的话,
在那之后,还会一次。也就是说,总共会有8次。现在的问题是,我今年多大了?”

\end{quote}

编写一个Python程序,搜索这个难题的答案。Hint:你或许会发现字符串方法{\tt zfill}对你有些帮助。

可以参看我的答案\url{thinkpython.com/code/cartalk.py}.

\end{ex}




























