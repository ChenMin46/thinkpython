\chapter{实例学习:数据结构的实例}

\section{单词频率分析}
\label{analysis}

按照惯例,在参考我的答案之前,最好尝试自己做下面的练习。

\begin{ex}
编写程序,读取文件,把每行分解为一个个单词,并从单词中去除掉空白和标点符号
。然后把单词转换成大写形式。

\index{string module string模块}
\index{module!string}

Hint:{\tt string}模块提供{\tt whitespace}字符串,包含了空格,制表,换行等等
字符,{\tt punctuation}字符串包含标点符号。让我们看看,是否可以让Python
“诅咒”:

\beforeverb
\begin{verbatim}
>>> import string
>>> print string.punctuation
!"#$%&'()*+,-./:;<=>?@[\]^_`{|}~
\end{verbatim}
\afterverb

并且,可能会使用字符串方法{\tt strip},{\tt replace}和{\tt translate}.

\index{strip method strip方法}
\index{method!strip}
\index{replace method replace 方法}
\index{method!strip}
\index{translate method translate方法}
\index{method!translate}

\end{ex}



\begin{ex}
\index{Project Gutenberg}

打开古腾堡工程的首页(\url{gutenberg.net}),下载你最喜欢的过了版权期的文本书籍。

\index{plain text 纯文本文件}
\index{text!plain 完全的}

修改程序,读取下载的书籍,略过文件开头的头信息,像以前一样处理剩余的单词。

然后修改程序,统计书中单词的数目,和每个单词出现的次数。

\index{word frequency 单词频率}
\index{frequency!word 单词}

输出书中不同单词的数目。 比较不同时期,不同作者的书籍。哪个作者使用的词汇量
最大?
\end{ex}

\begin{ex}
修改前一个练习的程序,输出书中使用频率最大的20个单词。
\end{ex}


\begin{ex}
修改前面程序,读取单词列表(参看\ref{wordlist}部分),输出书中所有不在单词列表
的单词。他们中有多少是排版错误?有多少是常见的单词,有多少是生僻单词?
\end{ex}

\section{随机数字}

\index{random number 随机数字}
\index{number 数字,random 随机}
\index{deterministic}
\index{pseudorandom  伪随机}

给定同样的输入,大多数的计算机程序每次产生同样的输出,所以,他们是确定的。
确定性通常是一件好事儿,因为我们希望同样的计算产生同样的结果。对于某些程序,
我们却希望产生不可确定的结果。游戏就是一个很显然的例子,但是远远不止这一个例子。

编写一个真正具有不确定性的程序不是一件容易的事儿。但是,有一些方法可以使它看上
去是不确定的。一种是使用产生伪随机数算法。伪随机数字并不是真正的随即,因为
他们是通过确定的计算产生的,但是如果仅仅观察数字,是不可能把他们和真正随机数区分开来。


\index{random module random模块}
\index{module!random}

{\tt random}模块提供了可以产生伪随机数的函数(我们一下都用“随机数”来代称)。

\index{random function random函数}
\index{function!random}


{\tt random}函数返回一个从0.0到1.0的随机浮点数(包括0.0,但是不包括1.0)。每次
,调用{\tt random},就会的到一个随机数。看下面的循环:

\beforeverb
\begin{verbatim}
import random

for i in range(10):
    x = random.random()
    print x
\end{verbatim}
\afterverb

{\tt randint}函数接受一个{\tt low}和{\tt high}参数,返回{\tt low}和{\tt
	high}之间的一个整数(包括两者)。


\index{randint function randint函数}
\index{function!randint}

\beforeverb
\begin{verbatim}
>>> random.randint(5, 10)
5
>>> random.randint(5, 10)
9
\end{verbatim}
\afterverb

要想随机地从一个序列中选取元素,可以使用{\tt choice}:

\index{choice function choice函数}
\index{fucntion!choice}

\beforeverb
\begin{verbatim}
>>> t = [1, 2, 3]
>>> random.choice(t)
2
>>> random.choice(t)
3
\end{verbatim}
\afterverb

{\tt random}模块还提供了从连续分布中产生随机数函数,包括,高斯函数,指数函数,
$\gamma$ 分布,等等。

\begin{ex}
\index{histogram!random choice}

 编写函数\verb"choose_from_hist",接受一个直方图(在\ref{histogram}部分定义的),
 然后从直方图里随机返回一个数,和对应的频率。比如:

 \beforeverb
\begin{verbatim}
>>> t = ['a', 'a', 'b']
>>> h = histogram(t)
>>> print h
{'a': 2, 'b': 1}
\end{verbatim}
\afterverb

函数必须是{\tt 'a'}及频率$2/3$,和\verb"'b'"及频率$1/3$。

\end{ex}



\section{单词频率直方图}

下面的程序读取文件,建立文件中单词的频率直方图。

\index{histogram!word frequencies}

\beforeverb
\begin{verbatim}
import string

def process_file(filename):
    h = dict()
    fp = open(filename)
    for line in fp:
        process_line(line, h)
    return h

def process_line(line, h):
    line = line.replace('-', ' ')
    
    for word in line.split():
        word = word.strip(string.punctuation + string.whitespace)
        word = word.lower()

        h[word] = h.get(word, 0) + 1

hist = process_file('emma.txt')
\end{verbatim}
\afterverb
%

程序读取{\tt emma.txt}---是简.奥斯汀的小说《爱玛》。

\index{Austin,Jane}

\verb"process_file"循环遍历获取文件的每行,然后传递给\verb"process_line"。
直方图{\tt h}作为累加器使用。

\index{accumulator!histogram}
\index{traversal}

\verb"process_line"先使用字符串方法{\tt replace}把连字符替换为空格,然后使用
{\tt split}把每行分隔为字符串列表。接着,遍历单词列表,使用{\tt strip}和
{\tt lower}剔除标点并把字符串转换为小写形式。(这里说“转换“是简略的说法。记得
		我们说过字符串是不可变的,所以方法{\tt strip}和{\tt
		lower}等返回新的字符串。)

最后,\verb"process_line"通过创建新的项或者增加已存在项的值,更新直方图,

\index{update!histogram}

要计算文件中单词的总数,可以计算直方图的频数之和:

\beforeverb
\begin{verbatim}
def total_words(h):
    return sum(h.values())
\end{verbatim}
\afterverb
%

单词量等于字典项的数目:

\beforeverb
\begin{verbatim}
def different_words(h):
    return len(h)
\end{verbatim}
\afterverb
%

下面是输出结果的代码:

\beforeverb
\begin{verbatim}
print 'Total number of words:', total_words(hist)
print 'Number of different words:', different_words(hist)
\end{verbatim}
\afterverb
%

结果为:

\beforeverb
\begin{verbatim}
Total number of words: 161073
Number of different words: 7212
\end{verbatim}
\afterverb
%

\section{最常用单词}

\index{DSU pattern}
\index{pattern!DSU}

若要找出最常用的单词,我们可以应用DSU模式;\verb"most_common"接受一个
直方图,返回(单词-频数)列表,频数由小到大顺序排序。

\beforeverb
\begin{verbatim}
def most_common(h):
    t = []
    for key, value in h.items():
        t.append((value, key))

    t.sort(reverse=True)
    return t
\end{verbatim}
\afterverb
%

下面是通过循环实现输出最常用的十个单词代码:

\beforeverb
\begin{verbatim}
t = most_common(hist)
print 'The most common words are:'
for freq, word in t[0:10]:
    print word, '\t', freq
\end{verbatim}
\afterverb
%

《爱玛》中,最常用的十个单词是:

\beforeverb
\begin{verbatim}
The most common words are:
to      5242
the     5204
and     4897
of      4293
i       3191
a       3130
it      2529
her     2483
was     2400
she     2364
\end{verbatim}
\afterverb
%

\section{可变参数}
\index{optional parameter 可变参数}
\index{parameter!optional 可选的}

我们已经遇到接受可变参数\footnote{译注:英文为optional argument,直译为可选择的参数
,英文侧重于每一个参数,而中文翻译侧重为参数整体}的内置函数和方法。我们也是可以自定义带有可变参数的函数。比如,下面是输出直方图中最常见单词的函数:

\beforeverb
\begin{verbatim}
def print_most_common(hist, num=10)
    t = most_common(hist)
    print 'The most common words are:'
    for freq, word in t[0:num]:
        print word, '\t', freq
\end{verbatim}
\afterverb

第一个参数是必须的;第二个参数是可选的。{\tt num}的缺省值是10。

\index{default value 缺省值}
\index{value!default 缺省}

如果只提供一个参数:

\beforeverb
\begin{verbatim}
print_most_common(hist)
\end{verbatim}
\afterverb

{\tt num}就是缺省值。如果提供两个参数:

\beforeverb
\begin{verbatim}
print_most_common(hist, 20)
\end{verbatim}
\afterverb

{\tt num}使用传递来的参数。换句话说,可选参数覆盖了缺省值。

\index{override 覆盖}

如果函数拥有可变参数,所有必需参数必须首先赋值,最后才能是可选参数。


\section{字典减法}
\index{dictionary!subtraction}
\index{subtraction!dictionary}

从书中找出不在单词列表{\tt words.txt}的单词,可以看做是集合减法运算;也就是
说,我们从一个集合(书中单词)找出不在另一个集合的单词(单词表里)。

{\tt subtract}接受{\tt d1}和{\tt d2}两个字典,返回一个包含所有不在{\tt d2}里
的{\tt d1}中的单词为关键字构成的字典。这里,我们不必关心关键字值大小,所以
我们把关键字值设为None。

\beforeverb
\begin{verbatim}
def subtract(d1, d2):
    res = dict()
    for key in d1:
        if key not in d2:
            res[key] = None
    return res
\end{verbatim}
\afterverb

发现书中不在{\tt words.txt}中单词,我们可以使用\verb"process_file"为{\tt
	words.txt}
	建立一个直方图,然后相减:

\beforeverb
\begin{verbatim}
words = process_file('words.txt')
diff = subtract(hist, words)

print "The words in the book that aren't in the word list are:"
for word in diff.keys():
    print word,
\end{verbatim}
\afterverb

下面是处理《爱玛》后的结果:

\beforeverb
\begin{verbatim}
The words in the book that aren't in the word list are:
 rencontre jane's blanche woodhouses disingenuousness 
friend's venice apartment ...
\end{verbatim}
\afterverb

有些单词是事物(人,动物,地名)名字。有些,像“rencontre"不是很常见。但是
有些常见的单词没有包含在单词列表中。

\begin{ex}

\index{set 集合}
\index{type!set 集合}

Python提供了{\tt
	set}数据结构,拥有一些常见的集合原算。阅读官方文档(\url{docs.python.org/lib/types-set.html}),
	编写程序使用集合减法,查找书中不在单词列表里的单词。

\end{ex}



\section{随机单词}
\label{randomwords}


\index{histogram!random choice}

要从直方图中随机选取一个单词,最简单的算法是依据每个单词的频数,创建一个列表
,列表中每个单词出现的次数等于频数。

\beforeverb
\begin{verbatim}
def random_word(h):
    t = []
    for word, freq in h.items():
        t.extend([word] * freq)

    return random.choice(t)
\end{verbatim}
\afterverb
%



表达式{\tt [word] * freq}创建一个列表,列表包含了{\tt freq}个字符串{\tt
	word}。
{\tt extend}方式和{\tt append}方法类似,除了它的参数是一个序列。


\begin{ex}
\label{randhist}

\index{algorithm 算法}


这个算法是有效的,但是不是很高效。每次,随机选择一个单词,就会重新创建一个
列表(近乎和原书一样大小)。一个显著的改进方案是只创建一次列表,然后
多次选择,然而列表还是很大。

一个替代方案:

\begin{enumerate}

\item 使用{\tt keys}获取书中单词列表。

\item 创建一个包含单词频数累计和的列表(参看练习\ref{cumulative})。最后一项
是书中单词的总数目,$n$。

\item 随机选择1到$n$,内的任意整数。使用二分搜索(参看练习\ref{bisection}),查找
插入随机数的索引。

\item 使用索引查找单词列表里对应的单词。

\end{enumerate}

编写程序使用此算法从书中随机生成一个单词。

\end{ex}


\section{马尔可夫分析}

\index{Markov analysis 马尔可夫分析}

 如果从书中随机选择单词,只能得到词汇,但不能得到句子。

 \beforeverb
\begin{verbatim}
this the small regard harriet which knightley's it most things
\end{verbatim}
\afterverb
%

连续的随机单词没有什么意义,因为这些连续的单词之间并没有什么语义上的关系。
比如,在真正的句子中,“the”后面通常跟着形容词或名词,不会跟着动词或副词。

检测这些联系的方法之一是马尔可夫分析\footnote{这个实例分析源自Kernighan and
	Pike, {\em The Practice of Programming}, 1999
		的一个例子},它刻划了,对于给定的一系列单词的下一个单词
		的可能性。比如,歌曲{\em Eric ,the Half a Bee}:

\begin{quote}
Half a bee, philosophically, \\
Must, ipso facto, half not be. \\
But half the bee has got to be \\
Vis a vis, its entity. D'you see? \\
\\
But can a bee be said to be \\
Or not to be an entire bee \\
When half the bee is not a bee \\
Due to some ancient injury? \\
\end{quote}

在歌词中,”bee“,总是跟在短语”half the“后面,但是短语”the bee“的后面跟着
”has“或“is“。

\index{prefix 前缀}
\index{suffix 后缀}
\index{mapping 映射}

马尔可夫分析的结果是从前缀(像“half the“和“the bee“)到所有可能的后缀(像
		“has“和“is“)的映射。


\index{random text }
\index{text!random}


给定这个映射,就可以产生一篇随即文本---从任何前缀开始,然后随机选取可能的
后缀。接着,可以结合当前前缀的后部分和后缀形成一个新的前缀,如此反复。

比如,从前缀“Half a“开始,下一个单词必须是“bee“,因为前缀仅仅在歌词中出现
一次。下一个前缀是“a bee“,所以下一个后缀可能是"philosophically","be"或者
"due“。

在这个例子中,前缀的长度总是为2,但是你可以使用任何长度的前缀。前缀的长度称为,
分析的“阶“。


\begin{ex}
马尔可夫分析:

\begin{enumerate}

\item
编写程序读取文档内容,然后进行马尔可夫分析。结果为从前缀到可能后缀
集合\footnote{译注:collection,不是set}的映射(字典)。集合可以是列表,元组或者字典,这个完全由你自己做主。可以使用长度为2的前缀测试程序,但是必须编写
程序使得很容易尝试其他长度的前缀。

\item 向之前的程序添加一个函数,依据马尔可夫分析随机生成一段文本。下面是一个
使用前缀长度为2的例子:

\begin{quote}
He was very clever, be it sweetness or be angry, ashamed or only
amused, at such a stroke. She had never thought of Hannah till you
were never meant for me?" "I cannot make speeches, Emma:" he soon cut
it all himself.
\end{quote}

对于这个例子,我把标点附加在单词后面。从句法上看,结果大多正确,但是不是全部符合规范。从语义上看,情况亦如此。

如果增加前缀的长度,情况会如何?随机生成的文本会更有意义吗?

\index{mash-up 混合}

\item 一旦程序可以工作,或许想尝试一下混合的效果:
如果分析两本或者更多的书籍,随机产生的文本将会以一种有趣的现象混合了不同
作者的词汇和短语。

\end{enumerate}
\end{ex}

\section{数据结构}

\index{data structure 数据结构}

使用马尔可夫分析生成随机文本非常有意思,但是这个练习有一个要点:数据结构的
选择。前面的练习,必须选择:

\begin{itemize}

\item 如何表示前缀。
\item 如何表示可能后缀的集合\footnote{译注:collection}。
\item 如何表示从前缀到后缀集合的映射。
\end{itemize}

恩,最后一个是很简单的,我们至今学到的映射数据类型是字典,所以自然选择它了。

对于前缀,最明显的选择就是使用字符串,字符串列表,或者字符串元组。对于后缀,
一种选择是列表,另外一种是直方图(字典)。

\index{implementation 实现}

应该如何选择数据结构呢?第一步,思考要为数据结构实现的操作。对于前缀,我们需要能够
从前部删除单词,添加到后部。比如,如果当前的前缀是"Half a“,下面一个单词是
"bee",需要能够形成下一个前缀“a bee“。

\index{tuple!as key in dictionary}

 第一个想到的可能是列表,因为列表很容易实现添加和删除元素,但是我们需要能够
 把前缀作为字典的关键字使用,所以排除列表。对于元组,不能实现添加和删除,但是
 可以使用“+”运算符构建新的元组。

 \beforeverb
\begin{verbatim}
def shift(prefix, word):
    return prefix[1:] + (word,)
\end{verbatim}
\afterverb
%

{\tt shift}接受单词元组,{\tt prefix}和一个字符串{\tt word},创建一个新的元组,
包含{\tt prefix}中除第一个以外的全部单词,{\tt word}被添加到新元组的尾部。

对于后缀集合,我们需要的操作包括添加一个新后缀(或者增加已存在后缀的频数),
和随机选择一个后缀。

添加一个新后缀对于列表实现或者直方图实现都是很容易的。从列表中随机选择一个
元素也是很容易的;但要从直方图中选择,就不是很高效(参看练习\ref{randhist}).。


迄今为止,我们已经谈论的大多是容易实现,但是还有其他的因素在选择数据结构时。
一种是运行时间。有时, 理论上,希望一种数据结构比其他的要快。比如,
我提到,{\tt in}操作符,至少在元素数目很大时,用于字典比列表要快。


 但是,通常我们之前并不知道哪种实现快?一种方式是实现两种,测试一下哪个更好。
 这种方式叫做基准程序。一个更实用的方法是选择最容易实现的数据结构,看看
 对于要实现的应用程序是否足够快。如果是,没有必要再实现其它数据结构。如果不是
 ,有其他的程序,像{\tt profile}模块,可以确定程序中花费时间最长的部分。

 \index{benchmarking 基准程序}
 \index{profile module profile模块}
 \index{module!profile}

另外一种需要考虑的因素是存储空间。比如,使用直方图存储后缀集合需要极小
的空间,因为只需要存储每个单词一次,无论它在文本中出现多少次。某种情况下,节省
存储空间安也可使得运行速度变快。极端地来说,如果你使用完内存,程序根本
无法运行。但是,对于很多应用程序,空间在运行后是次要考虑因素。

最后思考一下:此探讨中,我隐示地说明,在分析和生成阶段使用同一个数据结构。
但是,因为是分离的阶段,所以,可能使用一种数据结构分析,然后转换为另外一种数据结构生成。如果生成的时间超过转换的时间,这种方式也是不错的选择。



\section{调试}
\index{debugging 调试}

当调试程序时,特别是调试一个极难“对付”的bug时,我们可以尝试下面的四种方法:

\begin{description}

\item [阅读:]重新检查代码,确保程序表达了自己的意愿。

\item [运行:]修改程序,运行不同版本的程序。通常,如果在适当的位置,输出
了预计的东西,问题就很明显,但可能有时候,需要花些时间构建脚手架。

\item [反思:]花些时间反思!到底是出现了什么错误:语法,运行时,还是语义
粗我?可以从错误信息中或者程序输出得到什么信息?什么错误可能导致遇到错误?
问题出现前,最后一次改动是什么?

\item [“退后”:]有时候,最好的办法就是恢复改变,直到程序能工作,并且很
容易理解。然后,就可以重建构建程序。

\end{description}

初级程序员有时只专注于一种方法,忽略了其他的方法。每种方法都有失效的时候。

\index{typographical error  印刷错误}

比如,如果问题粗出在印刷错误上,阅读程序会有帮助,但如果是概念性误解,就
毫无帮助了。如果自己都不理解自己的程序,就算是读上百遍,也看不到什么错误。
因为错误在大脑里。

\index{experimental debugging 实验性调试}

做实验是很有效的,特别是运行简单的测试。但是,如果在实验时,不加思考,不阅读
自己的程序,很可能会落入“瞎猫碰死耗子编程”模式\footnote{译注:random walk 
	programming}---随意做出改变,直到程序得出正确的结果的过程。
	毋需说,“瞎猫碰死耗子编程”是费时,费力的。

\index{random walk programming 瞎猫碰死耗子编程}
\index{development plan!random walk programming}

必须花费时间思考。调试就像做科学实验。必须首先假设出现了什么问题。如果有
两种或多种可能性,那就测试一下,排除一个。


适当的休息一下,有助于思考。探讨也是。如果向其他人(甚至是自己)解释问题,
有时就会在在问问题的过程中,得到答案。

但是,如果程序有太多的错误,或者要调试的代码太大,太复杂,甚至最好的调试
技巧都会失败。有时,最好的办法就是退后,简化程序,直到程序能够运行,并且
很容易理解(至少自己能够理解)。

初级程序员通常不情愿退后,因为,他们不能忍受删除一行代码(尽管它是错误的)。
如果实在心有不舍,把程序复制一份,然后退后。这样,随后就可以一点点的粘贴回来


发现一个隐藏的bug需要阅读,运行,反思,有时还需要退后。如果有一种方式行不通,
使用其它的方法。

\section{术语表}

\begin{description}

\item [deterministic 确定的:]给定相同的输入,程序每次运行时,做同样的事情,
给出同样的输出。
\index{deterministic 确定的}

\item [pseudorandom 伪随机:]由确定性程序生产,表面上是随机的数字序列。
\index{pseudorandom 伪随机}

\item [default value 缺省值:]如果没有参数提供,由可选参数提供的值。
\index{default value 缺省值}

\item [override 覆盖:]使用参数代替缺省值。
\index{override 覆盖}

\item [benchmarking
基准程序:]实现候选的数据结构,给定一定的输入进行测试的过程。

\index{benchmarking 基准程序}

\end{description}


\section{练习}

\begin{ex}

\index{word frequency 单词频数}
\index{frequency!word}
\index{Zipf's law}

单词的“等级”就是在依照频数排序的单词列表里的位置:最常见的单词是得等级1,此常见
单词是等级2,等等。

Zipf法则用自然语言描述了单词等级和频数之间的关系\footnote{参看\url{wikipedia.org/wiki/Zipf's_law}.}。特别地,它预测了等级为$r$的单词的$f$频数:

\[ f = c r^{-s} \]
$s$和$c$是依赖于语言和文本的参数。如果,对等式两边取对数,得到:

\index{logarithm 对数}

\[ \log f = \log c - s \log r \]

如果绘制$\log f$--$\log r$,图像,就会得到一条斜率为$-s$,截距为$\log c$的
直线。

编写程序,读取文件,计算单词频数,按频数降序排列,每行打印一个单词,和
对应$\log f$和$\log r$。使用绘图程序绘制结果,并且检查看看是否形成直线。
你能够估测$s$的值吗?
\end{ex}















 











































































































































































