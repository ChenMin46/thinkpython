\chapter{调试}

\index{调试}

程序中会出现不同的错误,加以区分有助于加快对错误的跟踪。

\begin{itemize}

\item 语法错误是Python在将源代码转换为字节码时产生的。它们通常说明程序的语法有错误。例如:在 {\tt def}语句后忽略冒号会产生类似{\tt SyntaxError: invalid syntax}的信息。

\item 运行时错误由解释器在程序运行出错时产生。大多数的运行时错误消息包含错误发生的位置和正在执行的函数。例如,一个无穷递归的函数最终导致运行时错误“maximum recursion depth exceeded”。

\item 语义错误指程序执行过程中没有产生出错信息,但程序没有做正确的工作。例如,一个表达式没有按照你期望的顺序进行求值,导致错误的结果。

\end{itemize}

\index{语法错误}
\index{运行时错误}
\index{语义错误}
\index{错误!编译时}
\index{错误!语法}
\index{错误!运行时}
\index{错误!语义}
\index{异常}

调试的第一步是找出你遇到了什么类型的错误。虽然下面的章节根据错误类型组织的,但一些方法使用于多种情况。


\section{语法错误}

\index{错误消息}

当你找到原因后语法错误一般很容易修正。不幸的是,错误消息常常不是很有帮助。最常见的消息是{\tt SyntaxError: invalid syntax}和{\tt SyntaxError: invalid token},它们都不包含很多信息量。

另一方面,错误消息告诉你问题发生在程序中的什么位置。事实上,当Python遇到问题时它将告诉你,但这不一定是错误的地方。有时错误在错误消息位置的前面,通常是前几行。

\index{增量开发}
\index{开发计划!增量}

如果你增量的开发程序,你会清楚错误在哪里。它就是你最新添加的代码。

如果你从书中复制一段代码,那么从仔细地检查你的代码和书中的代码开始。检查每一个字符。同时记住书本也会有错误,如果你遇到类似语法错误,也许它就是。

下面给出一些避免常见语法错误的方法:

\index{syntax}

\begin{enumerate}

\item 避免使用Python关键字作为变量名。

\index{关键字}

\item 确保你在每个复合语句头的尾部加上了冒号,包括{\tt for},{\tt while},{\tt if}和{\tt def}。

\index{头部}
\index{冒号}

\item 确保代码中的字符串都有匹配的引号。

\index{引号标记}

\item 如果你用三重引号标记多行字符串,确保你正确的结束该字符串。一个未终止的字符串会导致程序尾部{\tt invalid token}错误,或者程序接下来的部分都被认为字符串,直到下一个字符串。在第二种情况,Python可能不会产生一个错误消息!

\index{多行}
\index{字符创!多行}

\item Python将未封闭的运算符---\verb+(+,\verb+{+,或\verb+[+---的下一行作为当前语句的一部分。通常第二行将产生错误。

\item 在条件语句中使用传统的{\tt =}而不是{\tt ==}。

\index{条件}

\item 确保每行缩进正确。Python可以处理空格和tabs,但如果你混淆它们会产生错误。避免这个问题最好的方法是使用适合Python编辑、会自动缩进的文本编辑器。

\index{缩进}
\index{空白符}

\end{enumerate}

如果这些没有效,继续阅读下一章节...


\subsection{我做了修改但是没有任何区别}

如果解释器报错,但你找不到错误,有可能你和解释器看到不不是相同的代码。检查你的编程环境,确保你正在编辑的文件是Python将要运行的。

如果你不确定,在程序的开始位置添加明显的故意的语法错误,再次运行,如果解释器没有找到这个错误,说明你不在运行新的程序。

下面是一些可能的出错原因:

\begin{itemize}

\item 你编辑了文件,运行前忘记了保存。有的编程环境会为你自动保存,有的不会。

\item 你修改了文件的名字,但仍然运行老的名字。

\item 你的开发环境配置有误。

\item 如果你在编写模块并使用{\tt import},确保你写的模块名不同于Python标准模块名。

\index{模块!reload}
\index{reload函数}
\index{函数!reload}

\item 如果你使用{\tt import}读取一个模块,记得重启解释器或使用{\tt reload}来读取一个修改过的文件。如果你再次导入模块,解释器将不做任何事。

\end{itemize}

如果你陷入困境找不到出错的原因,一个方法是从一个类似“Hello,World!”的新程序开始,确保你从一个已知的可运行的程序开始。然后逐步添加原程序中的代码到新程序。


\section{运行时错误}

当你的程序没有语法错误,Python可以编译并运行。这时可能发生什么错误呢?


\subsection{我的程序什么也没做}

这个问题通常发生在你的文件包含函数和类,但没有任何调用执行的语句。也许你故意这样,因为你仅仅打算导入这个模块来提供函数和类。

如果这不是故意的,确保你调用一个函数来开始执行,或从一个交互的命令提示行下执行。参见下面的“执行流程”章节。


\subsection{我的程序挂起了}
\index{无限循环}
\index{无限递归}
\index{挂起}

如果一个程序停止,但上去什么也没做,我们称“挂起”。通常意味着程序陷入一个无限循环或者无线递归。

\begin{itemize}

\item 如果你怀疑某个循环引起这个问题,在循环开始时添加{\tt print}语句打印“进入循环”,在循环结束时打印“exiting the loop”退出循环。

运行程序,如果你的到第一条消息,但没有第二条消息,你得到一个无限循环。参见“无限循环”章节。

\item 大多数时候,无限循环会令程序运行一段时间,然后产生“RuntimeError: Maximum recursion depth exceeded”的错误。如果是这样,参考下面的“无限递归”章节。

如果你没有得到这样的错误信息,但你怀疑某个递归方法或函数有问题,你仍可以使用“无限递归”章节中提到的方法。

\item 如果这些方法都没有,尝试测试其他的循环、递归的函数和方法。

\item 如果还是没有效果,有可能你不清楚程序执行的流程。参考下面的“执行流程”章节。

\end{itemize}


\subsubsection{无限循环}
\index{无限循环}
\index{循环!无限}
\index{条件}
\index{循环!条件}

如果你有一个无限循环并且你认为这个循环导致了问题,在循环体结束的地方添加{\tt print}语句打印条件语句中的变量值和条件值。

例如:

\beforeverb
\begin{verbatim}
while x > 0 and y < 0 :
    # do something to x
    # do something to y

    print  "x: ", x
    print  "y: ", y
    print  "condition: ", (x > 0 and y < 0)
\end{verbatim}
\afterverb
%
现在当你运行程序,每个循环你都会看到3行输出。对于最后一次循环,条件应该为{\tt false}。如果循环不断执行,你可以看到{\tt x}和{\tt y}的值,也许能够发现为什么它们没有被正确的更新。


\subsubsection{无限递归}
\index{无限递归}
\index{递归!无限}

大多数时候,无限循环会令程序运行一段时间,然后产生“RuntimeError: Maximum recursion depth exceeded”的错误。

如果你怀疑一个函数或方法导致了无限递归,首先检查存在一个基本状态。换言之,应该有一个状态使函数或方法直接返回,而不是再次递归调用。如果不是,你需要重新思考算法,并设计一个基本状态。

如果存在一个基本状态,但程序似乎没有运行到这个状态,你可以在函数或方法开始的部分添加{\tt print}语句打印参数。现在当你运行程序,每次函数或方法被调用时,你将看到几行关于参数的输出。如果参数没有向基本状态靠拢,你也许会发现为什么这样。


\subsubsection{执行流程}
\index{执行流程}

如果你不确定程序的执行流程,在每个函数开始的地方加入{\tt print}语句,打印类似“entering function {\tt foo}”的语句,其中{\tt foo}是函数的名字。

现在当你运行程序,它将打印每个函数被调用的追踪。


\subsection{当我运行程序,我得到一个异常}
\index{异常}
\index{运行时错误}

有时程序运行时出错,Python将打印异常的名字、问题发生的位置和回溯。

\index{回溯}

回溯识别当前运行的函数,然后识别调用该函数的函数,以此类推。换言之,它追踪了你到达这个错误所经过的函数调用。同时它也包括这些调用在文件中的位置。

第一步是检查代码中对应的出错的位置,或许你能发现错误。下面是常见的运行时错误:

\begin{description}

\item[名字错误:]  你试图使用一个不在当前环境下的变量。记住局部变量是局部的,你不能在定义它们的函数的外部引用它们。

\index{名字错误}
\index{类型错误}
\index{异常!名字错误}
\index{异常!类型错误}

\item[类型错误:] 可能的几个原因是:

\begin{itemize}

\item 你试图不适当的使用值。例如:使用一个非整数的值作为字符串、列表或元组的下标。

\index{下标}

\item 格式字符串和用于转换的对象不匹配。这发生在或者数量不相同,或者调用了一个非法的转换。

\index{格式运算符}
\index{运算符!格式}

\item 你调用函数或方法时传递的参数个数有误。对于方法,检查方法定义,确保第一个参数是{\tt self},然后检查方法调用,确保你对一个对象调用方法,并正确的提供其他参数。

\end{itemize}

\item[键错误:]  你试图使用字典中不存在的键访问对应的元素。

\index{键错误}
\index{异常!键错误}
\index{字典}

\item[属性错误:] 你试图访问一个不存在的属性或方法。检查拼写!你可以使用{\tt dir}来列举存在的属性。

如果属性错误指出一个对象是{\tt 空类型},及它是{\tt 空}的。一个常见的原因是函数结束时忘记返回值。如果在返回尾部没有{\tt return}语句,函数将返回一个{\tt None}。另一个原因是使用类似列表中的{\tt sort}方法,它返回{\tt None}。

\index{属性错误}
\index{异常!属性错误}

\item[下标错误:] 用来访问列表、字符串或元组的下标超过长度减1。在错误发生的前一行使用{\tt print}语句显示下标值和序列的长度,检查两个值是否正确。

\index{下标错误}
\index{异常!下标错误}

\end{description}

\index{调试器(pdb)}
\index{Python调试器(pdb)}
\index{pdb(Python调试器)}

Python调试器({\tt pdb})允许你在错误前检查程序状态,有助于追踪异常。你可以在 \url{docs.python.org/lib/module-pdb.html}阅读有关{\tt pdb}的内容。


\subsection{我添加了太多的{\tt print}语句,输出令我应接不暇}

\index{print语句}
\index{语句!print}

使用{\tt print}语句调试的一个问题是你会被大量的输出信息掩埋。有两个处理方法:简化输出或简化程序。

简化输出可以删除或注释不用的{\tt print}语句,或将它们合并,或格式化输出使得它们容易理解。

简化程序你可以做一下几件事。首先缩小程序处理的问题的规模。例如,如果你在搜索一个列表,搜索一个{\em 小的}列表。如果程序从用户读取输入,输入最简单的参数。

\index{死区代码}

第二,整理程序,删除死区代码,使程序易读。例如,如果你认为问题深嵌在程序中,试图用简单的结构重写那部分。如果你怀疑一个大函数有错误,试图将它分割成小函数,然后分别测试。

\index{测试!最小测试案例}
\index{测试案例,最小}

通常寻找最小测试案例的过程让你找到错误。如果你发现一个程序对于一个情况适用,对另一个情况不使用,这将给你一些线索。

同样,重写一段代码有助于发现一些微小的错我。如果你做了一个你认为不会影响程序的修改,但事实上影响了,这个方法可以给你警戒。


\section{语义错误}
\index{语义错误}
\index{错误!语义}

在某种程度上,语义错误时最难调试的,因为解释器不能提供任何错误信息。你所知道的只有程序应该怎么做。

第一步是建立程序代码和你见到的行为之间的联系。你需要假设程序实际上做了什么。一个主要困难在于计算器运行的太快了。

你常希望你可以降低程序的速度,是人类可以跟上,通过使用调试器,你可以实现这步。但是在关键地方加入一些{\tt print}语句所用的时间通常短于建立调试器,加入删除断点,然后“逐步”运行程序直到错误发生。

\subsection{我的程序不工作}

你需要问自己这些问题:

\begin{itemize}

\item 有没有什么程序应该做却没有发生?找到执行该函数的代码段,确保程序被执行。

\item 有没有什么不应该发生的发生了?找到执行该函数的代码段,查看它是否执行了?

\item 有没有代码的执行效果与你期望的不同?确保你理解有问题的代码,尤其是包含调用其他Python模块中的函数或方法。阅读你调用的函数的文档,用一些简单的例子进行测试。

\end{itemize}

在编程时,你心中需要有一个关于程序如何工作的模型。如果你的程序没有按照你期望的工作,很可能问题不在于程序,而在于你心中的模型。

\index{模型,心中的}
\index{心中的模型}

修正你心中的模型的最好的方法是将程序分割为不同部分(通常是函数和方法),并分别测试。一旦你发现了模型和现实的差异,你就可以解决问题。

当然,在开发的过程中尼需要建立并测试组件。如果你遇到了问题,只有一小部分新的代码是不确定正确性的。


\subsection{我写了一个很长的表达式,它没有按照我期望的工作}

\index{表达式!大而复杂}
\index{大而复杂的表达式}

编写复杂的表达式是合理的,如果它们可读。但是它们调试起来很困难。通常我们将一个复杂的表达式分割成一系列的临时变量的赋值。

例如:

\beforeverb
\begin{verbatim}
self.hands[i].addCard(self.hands[self.findNeighbor(i)].popCard())
\end{verbatim}
\afterverb
%
可以重写为:

\beforeverb
\begin{verbatim}
neighbor = self.findNeighbor(i)
pickedCard = self.hands[neighbor].popCard()
self.hands[i].addCard(pickedCard)
\end{verbatim}
\afterverb
%
这个明晰的版本更适于阅读,因为变量名提供了额外的文档,同时调试也更容易,因为你可以检查中间变量的类型和它们的值。

\index{临时变量}
\index{变量!临时}
\index{运算符优先级}
\index{优先级}

大的表达式的另一个问题是计算的顺序不一定是你期望的。例如,如果你将表达式$\frac{x}{2 \pi}$翻译为Python,你也许会写成:

\beforeverb
\begin{verbatim}
y = x / 2 * math.pi
\end{verbatim}
\afterverb
%
这是不正确的,因为乘法和除法有相同的优先级,因此是从左往右计算的,这个表达式计算的是$x \pi / 2$。

调试表达式的一个好的方法是添加括号,使得计算顺序简洁明了:

\beforeverb
\begin{verbatim}
 y = x / (2 * math.pi)
\end{verbatim}
\afterverb
%
当你不确定计算优先级是,使用括号。不仅程序将工作正常(按你的要求执行),同时也让那些没有记住优先级规则的人阅读起来更方便。


\subsection{我的函数或方法没有按照我期望的返回}
\index{return语句}
\index{语句!return}

如果你的{\tt return}语句包含一个复杂的表达式,你没有机会在返回前打印{\tt 返回}值。同样你可以使用临时变量。例如:对于

\beforeverb
\begin{verbatim}
return self.hands[i].removeMatches()
\end{verbatim}
\afterverb
%
你可以写成:

\beforeverb
\begin{verbatim}
count = self.hands[i].removeMatches()
return count
\end{verbatim}
\afterverb
%
现在你在返回前可以打印{\tt count}的值。


\subsection{我实在是卡住了,我需要帮助}

第一,尝试离开电脑几分钟。电脑辐射会对大脑产生影响,导致下列几种症状:

\begin{itemize}

\item 沮丧和愤怒

\index{沮丧}
\index{愤怒}
\index{调试!情绪反应}
\index{带情绪的调试}

\item 迷信的人认为“电脑讨厌我”,并神奇的相信“程序仅当我向后戴着帽子时才工作正常”。

\index{调试!迷信}
\index{迷信的调试}

\item 随机漫步编程(用各种可能的方法编程,并选择工作正常的那个)。

\index{随机漫步编程}
\index{开发计划!随机漫步编程}

\end{itemize}

如果你发现你有以上任意一种症状,站起来走一走。当你心绪平静时,思考一下程序。它是做什么的?什么肯能造成了这种行为?上次可以工作的程序是什么时候?下一步做什么?

有时找到一个错误很费时间。我常常在我离开电脑,让思维游荡的时候找到错误。一些找到错误最好的地方有火车上,浴室里,以及临睡前。


\subsection{不,我真的需要帮助}

即使最好的程序员也会卡住。有时你在一个程序上工作了太长的时间,因此你难以发现错误。而他人可能一眼就发现问题。

在你向其他人寻求帮助前,你需要做好准备。你的程序需要尽可能简洁,你需要最少的输入来重现错误。你需要在合适的位置加入{\tt print}语句,同时输出应可理解。你需要能够以简洁的语言描述问题。

当你想某人求助,你需要提供足够的信息:

\begin{itemize}

\item 是否有出错消息?它是什么?指向程序的哪部分?

\item 错误出现前你做的最后一步是什么?你写的最后几行是什么?什么新的测试导致了错误?

\item 你做了哪些尝试?你学到了什么?

\end{itemize}

当你找到了错误,花时间想一想你怎么能更快的定位它。下一次你遇到类似的问题,你就可以更快的找到问题。

记住,目标不仅仅是让程序工作,而是学会如何让程序工作。

\printindex

\clearemptydoublepage
